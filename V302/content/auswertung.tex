\section{Auswertung}
\label{sec:Auswertung}
Die Messergebnisse der Wheatstoneschen Brückenschaltung für den Wert 11 sind in
folgender Tabelle dargestellt. $R_x$ berechnet sich nach \eqref{eqn:wheatstone}.

\begin{table}
  \centering
  \caption{Die Messwerte der Wheatstoneschen Brückenschaltung.}
  \label{tab:ausw:a}
  \sisetup{table-format=3.0}
  \begin{tabular}{S S S[table-format=3.2] S[table-format=4.0]}
    \toprule
    ${R_3 [\si{\ohm}]}$ &   ${R_4 [\si{\ohm}]}$ &   ${R_x [\si{\ohm}]}$ &   ${R_2 [\si{\ohm}]}$ \\
    \midrule
     333 &     667 &  499.25 &    1000 \\
     601 &     399 &  500.08 &     332 \\
    \bottomrule
  \end{tabular}
\end{table}

\begin{table}
  \centering
  \caption{Die Messwerte der Kapazitätsmessbrücke.}
  \label{tab:ausw:b}
  \sisetup{table-format=3.0}
  \begin{tabular}{S S S S S[table-format=3.2] S[table-format=3.2]}
  \toprule
     ${R_3 [\si{\ohm}]}$ &   ${R_4 [\si{\ohm}]}$ &   ${R_2 [\si{\ohm}]}$ &   ${C_2}$ &   ${R_x [\si{\ohm}]}$ &   ${C_x}$ \\
  \midrule
       411 &     589 &     684 &     450 &  477.29 &  644.89 \\
       605 &     395 &     316 &     992 &  487.06 &  647.67 \\
  \bottomrule
  \end{tabular}
\end{table}

\begin{table}
  \centering
  \caption{Die Messwerte der Induktivitätsmessbrücke.}
  \label{tab:ausw:c}
  \sisetup{table-format=3.0}
  \begin{tabular}{S S S S[table-format=2.1] S[table-format=3.2] S[table-format=2.2]}
  \toprule
     ${R_3 [\si{\ohm}]}$ &   ${R_4 [\si{\ohm}]}$ &   ${R_2 [\si{\ohm}]}$ &   ${L_2}$ &   ${R_x [\si{\ohm}]}$ &   ${L_x}$ \\
  \midrule
       403 &     597 &     722 &    20.1 &  487.38 &   13.57 \\
       408 &     592 &     700 &    20.1 &  482.43 &   13.85 \\
  \bottomrule
  \end{tabular}
\end{table}

\begin{table}
  \centering
  \caption{Die Messwerte der Maxwell-Brücke.}
  \label{tab:ausw:d}
  \sisetup{table-format=3.0}
  \begin{tabular}{S S S[table-format=4.0] S S[table-format=3.2] S}
    \hline
   ${R_3 [\si{\ohm}]}$ &   ${R_4 [\si{\ohm}]}$ &   ${R_2 [\si{\ohm}]}$ &   ${C_d}$ &   ${R_x [\si{\ohm}]}$ &   ${L_x}$ \\
   \hline
     340 &     307 &    1000 &     450 &  373.37 &     nan \\
     115 &     308 &    1000 &     450 &  373.38 &     nan \\
     \hline
  \end{tabular}
\end{table}

\begin{table}
  \centering
  \caption{Die Messwerte der Wien-Robinson-Brücke.}
  \label{tab:ausw:e}
  \sisetup{table-format=5.0}
  \begin{tabular}{S S[table-format=1.2]}
  \toprule
   ${f [\si{\hertz}]}$ &   ${U [\si{\volt}]}$ \\
   \midrule
         20 &     3     \\
        100 &     1     \\
        120 &     0.65  \\
        140 &     0.31  \\
        150 &     0.16  \\
        160 &     0.011 \\
        170 &     0.17  \\
        180 &     0.25  \\
        200 &     0.5   \\
        250 &     1     \\
        300 &     1.35  \\
        350 &     1.6   \\
        400 &     1.9   \\
        500 &     2.2   \\
        600 &     2.5   \\
        800 &     2.9   \\
       1000 &     3     \\
       2000 &     3.2   \\
       3000 &     3.3   \\
       4000 &     3.25  \\
       5000 &     3.3   \\
      10000 &     3.3   \\
      15000 &     3.2   \\
      20000 &     3.2   \\
    \bottomrule
  \end{tabular}
\end{table}
