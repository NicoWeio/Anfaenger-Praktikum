\section{Auswertung}
\label{sec:Auswertung}

\subsection{Die Wheatstonesche Brückenschaltung}
Die Messwerte und Ergebnisse der Wheatstoneschen Brückenschaltung für den Wert 11 sind in
folgender Tabelle dargestellt. $R_x$ berechnet sich nach \eqref{eqn:wheatstone}.
Der Koeffizient $\frac{R_3}{R_4}$ besitzt eine Unsicherheit von $\pm 0.5 \si{\percent}$.
Die Unsicherheit wurde nach der Gaußschen Fehlerfortpflanzung mittels \textit{uncertainties}
in \textit{python} berechnet.

\begin{table}[H]
  \centering
  \caption{Die Messwerte und Ergebnisse der Messung mittels Wheatstonescher Brückenschaltung.}
  \label{tab:ausw:a}
  \sisetup{table-format=3.0}
  \begin{tabular}{S[table-format=4.0] S S S[table-format=1.3] S[table-format=3.1]S[table-format=1.1]}
    \toprule
    ${R_2 [\si{\ohm}]}$ & ${R_3 [\si{\ohm}]}$ & ${R_4 [\si{\ohm}]}$ & ${\Delta \frac{R_3}{R_4}}$ & ${R_x [\si{\ohm}]}$ & ${\Delta R_x [\si{\ohm}]}$ \\
    \midrule
     1000 & 333 & 667 & 3.335 & 499.3  & 2.5 \\
      332 & 601 & 399 & 1.995 & 500.1  & 2.5 \\
    \bottomrule
  \end{tabular}
\end{table}
\noindent
Die Abweichungen von dem Literaturwert sind in folgender Tabelle wiedergegeben.

\begin{table}[H]
  \centering
  \caption{Die Abweichungen der Berechnung von Wert 11.}
  \label{tab:aus:aA}
  \sisetup{table-format=3.3}
  \begin{tabular}{S S S S}
    \toprule
    $R_{xLit} [\si{\ohm}]$ & $R_x [\si{\ohm}]$ & $\Delta R_x [\si{\ohm}]$ & ${\text{Abweichung} [\si{\percent}]}$ \\
    \midrule
    489.9 & 499.3 & 2.5 & 1.9 \\
    489.9 & 500.1 & 2.5 & 2.1 \\
    \bottomrule
  \end{tabular}
\end{table}
\noindent

\subsection{Die Kapazitätsmessbrücke}
Die Messwerte und die nach \eqref{eqn:Cbrücke} berechneten Ergebnisse sind in folgender Tabelle zusammengefasst. Bei der Messung wurde der Wert 15
verwendet. Der Widerstand $R_2$ besitzt eine Unsicherheit von $3 \si{\percent}$. Die Unsicherheiten wurden ebenfalls mit \textit{uncertainties}
berechnet.
\begin{table}[H]
  \centering
  \caption{Die Messwerte und Ergebnisse der Berechnung mittels Kapazitätsmessbrücke.}
  \label{tab:ausw:b}
  \sisetup{table-format=3.0}
  \begin{tabular}{S S[table-format=2.2] S S S S[table-format=3.2] S S[table-format=3.2]}
  \toprule
     ${R_2 [\si{\ohm}]}$ & ${\Delta R_2 [\si{\ohm}]}$ & ${R_3 [\si{\ohm}]}$ &  ${R_4 [\si{\ohm}]}$ & ${C_2 [\si{\nano\farad}]}$ & ${R_x [\si{\ohm}]}$ & ${\Delta R_x [\si{\ohm}]}$ & ${C_x [\si{\nano\farad}]}$ \\
  \midrule
       684 & 20.25 & 411 & 589 & 450 &  477.29 & 14 & 644.89 \\
       316 &  9.48 & 605 & 395 & 992 &  484.0  & 15 & 647.67 \\
  \bottomrule
  \end{tabular}
\end{table}
\noindent
Die Abweichungen der errechneten Werte von den Literaturwerten sind in folgender Tabelle wiedergegeben.

\begin{table}[H]
  \centering
  \caption{Die Abweichungen der Berechnung von Wert 15.}
  \label{tab:aus:bB}
  \sisetup{table-format=3.2}
  \begin{tabular}{S S S[table-format=1.2] S[table-format=3.0] S S[table-format=1.2]}
    \toprule
    $R_{xLit} [\si{\ohm}]$ & $R_x [\si{\ohm}]$ & ${\text{Abw} [\si{\percent}]}$ & $C_{xLit} [\si{\nano\farad}]$ & $C_x [\si{\nano\farad}]$ & ${\text{Abw} [\si{\percent}]}$ \\
    \midrule
    473 & 477.29  & 0.9  & 652 & 644.89 & 1.09 \\
    473 & 484.0   & 2.32 & 652 & 647.67 & 0.66 \\
    \bottomrule
  \end{tabular}
\end{table}
\noindent

\subsection{Die Induktivitätsmessbrücke}
Die Messwerte und die nach \eqref{eqn:Lbrücke} berechneten Ergebnisse sind in folgender Tabelle wiedergegeben.
Es wurde der Wert 18 verwendet.

\begin{table}[H]
  \centering
  \caption{Die Messwerte und Ergebnisse der Berechnung mittels Induktivitätsmessbrücke.}
  \label{tab:ausw:c}
  \sisetup{table-format=3.0}
  \begin{tabular}{S S S S[table-format=2.2] S[table-format=3.2] S[table-format=2.2]}
  \toprule
     ${R_2 [\si{\ohm}]}$ & ${R_3 [\si{\ohm}]}$ &  ${R_4 [\si{\ohm}]}$ & ${L_2 [\si{\milli\henry}]}$ & ${R_x [\si{\ohm}]}$ & ${L_x [\si{\milli\henry}]}$ \\
  \midrule
       722 & 403 & 597 & 20.1 &  487.38 &  13.57 \\
       700 & 408 & 592 & 20.1 &  482.43 &  13.85 \\
  \bottomrule
  \end{tabular}
\end{table}
\noindent
Die Abweichungen der errechneten Werte von den Literaturwerten sind in folgender Tabelle wiedergegeben.

\begin{table}[H]
  \centering
  \caption{Die Abweichungen der Berechnung von Wert 18.}
  \label{tab:aus:cC}
  \sisetup{table-format=3.2}
  \begin{tabular}{S[table-format=3.0] S S[table-format=2.2] S[table-format=2.2] S[table-format=2.2] S[table-format=2.2]}
    \toprule
    $R_{xLit} [\si{\ohm}]$ & $R_x [\si{\ohm}]$ & ${\text{Abw} [\si{\percent}]}$ & $L_{xLit} [\si{\milli\henry}]$ & $L_x [\si{\milli\henry}]$ & ${\text{Abw} [\si{\percent}]}$ \\
    \midrule
    360 & 487.38   & 35.38  & 49.82 & 13.57 & 72.77 \\
    360 & 482.43   & 34.0   & 49.82 & 13.85 & 72.19 \\
    \bottomrule
  \end{tabular}
\end{table}
\noindent

\subsection{Die Maxwellbrücke}
Die Messwerte und die nach \eqref{eqn:maxwell} berechneten Ergebnisse sind in folgender Tabelle wiedergegeben.
Hierbei besitzen die Widerstände $R_3$ und $R_4$ eine Unsicherheit von $\pm 3 \si{\percent}$. Die resultierenden
Unsicherheiten wurden hier ebenfalls mit \textit{uncertainties} berechnet.

\begin{table}[H]
  \centering
  \caption{Die Messwerte und Ergebnisse der Berechnung mittels Maxwellbrücke.}
  \label{tab:ausw:d}
  \sisetup{table-format=3.0}
  \begin{tabular}{S[table-format=4.0] S S[table-format=2.2] S S[table-format=1.2] S S[table-format=4.0] S[table-format=2.0] S[table-format=3.2] S[table-format=1.1]}
  \toprule
     ${R_2 [\si{\ohm}]}$ & ${R_3 [\si{\ohm}]}$ & ${\Delta R_3 [\si{\ohm}]}$ & ${R_4 [\si{\ohm}]}$ & ${\Delta R_4 [\si{\ohm}]}$ & ${C_4 [\si{\nano\farad}]}$ & ${R_x [\si{\ohm}]}$ & ${\Delta R_x [\si{\ohm}]}$ & ${L_x [\si{\milli\henry}]}$ & ${\Delta L_x [\si{\milli\henry}]}$ \\
  \midrule
       1000 & 340 & 10.2 & 307 & 9.21 & 450 &  1107 & 33 &  153.0 & 5 \\
       1000 & 115 & 3.45 & 308 & 9.24 & 450 &  373  & 11 &  51.8  & 1.6\\
  \bottomrule
  \end{tabular}
\end{table}
\noindent
Die Abweichungen der Werte von den Literaturwerten ist in folgender Tabelle angegeben.



\subsection{Die Wien-Robinson Brücke}

\begin{table}[H]
  \centering
  \caption{Die Messwerte der Wien-Robinson-Brücke.}
  \label{tab:ausw:e}
  \sisetup{table-format=5.0}
  \begin{tabular}{S S[table-format=1.2]}
  \toprule
   ${f [\si{\hertz}]}$ &   ${U [\si{\volt}]}$ \\
   \midrule
         20 &     3     \\
        100 &     1     \\
        120 &     0.65  \\
        140 &     0.31  \\
        150 &     0.16  \\
        160 &     0.011 \\
        170 &     0.17  \\
        180 &     0.25  \\
        200 &     0.5   \\
        250 &     1     \\
        300 &     1.35  \\
        350 &     1.6   \\
        400 &     1.9   \\
        500 &     2.2   \\
        600 &     2.5   \\
        800 &     2.9   \\
       1000 &     3     \\
       2000 &     3.2   \\
       3000 &     3.3   \\
       4000 &     3.25  \\
       5000 &     3.3   \\
      10000 &     3.3   \\
      15000 &     3.2   \\
      20000 &     3.2   \\
    \bottomrule
  \end{tabular}
\end{table}
