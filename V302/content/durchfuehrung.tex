\section{Durchführung}
\label{sec:Durchführung}
Alle in Kapitel \ref{sec:Theorie} beschriebenen Brückenschaltungen werden an eine Wechselstromquelle mit einer Frequenz von 
$\omega=\SI{1}{\kilo\hertz}$ angeschlossen. Für die Messung der Brückenspannung wird ein Oszilloskop verwendet. Der Aufbau der 
Schaltungen wurde in Kapitel \ref{sec:Theorie} ausführlich behandelt. 

\subsection{Wheatstone'sche Brückenschaltung}
\label{sec:durch-wheatstone}
Bei der Messung mit der Wheatstone'schen Brückenschaltung (vgl. \ref{sec:wheatstone}) wird ein unbekannter Widerstand ausgemessen.
Das Potentiometer $R_3/R_4$ dafür so lange variiert, bis die gemessene Brückenspannung gleich null ist. 
Die Werte der bekannten Widerstände $R_2$, $R_3$ und $R_4$ werden abgelesen notiert.
Diese Messung wird für zwei verschiedene Widerstände $R_2$ durchgeführt.
Falls Hochfrequente Störspannungen die Messung stören, können diese mit einem Tiefpass unterdrückt werden. 

\subsection{Kapazitätsmessbrücke}
\label{sec:durch-Cbrücke}
Bei der Messung mit der Kapazitätsmessbrücke (vgl. \ref{sec:Cbrücke}) wird ein unbekannter Kondensator ausgemessen. 
Die Stellglieder $R_2$ und Potentiometer $R_3/R_4$ werden dafür so lange variiert, bis die Brückenspannung gleich null ist.
Die Werte der bekannten Widerstände $R_3$ und $R_4$ und der Kapazität $C_2$ werden abgelesen und notiert. 
Diese Messung wird für zwei verschiedene Kondensatoren $C_2$ durchgeführt.
Anschließend wird die Messung für ein $RC$-Element wiederholt.

\subsection{Induktivitätsmessbrücke}
\label{sec:durch-Lbrücke}
Bei der Messung mit der Kapazitätsmessbrücke (vgl. \ref{sec:Lbrücke}) wird eine unbekannte Induktivität ausgemessen. 
Die Stellglieder $R_2$ und Potentiometer $R_3/R_4$ werden dafür so lange variiert, bis die Brückenspannung gleich null ist.
Die Werte der bekannten Widerstände $R_3$ und $R_4$ und der Induktivität $L_2$ werden abgelesen und notiert.
Diese Messung wird für zwei verschiedene Widerstände $R_2$ durchgeführt.

\subsection{Maxwell}
\label{sec:durch-maxwell}
Die beiden Induktivitäten, die mit der Induktivitätsmessbrücke ausgemssen wurden, werden nun erneut mit der Maxwell-Brücke 
(vgl. \ref{sec:maxwell}) ausgemessen. $R_3$ und $R_4$ sind dabei nach Abbildung \ref{fig:maxwell} nicht mehr als Potentiometer
verbaut, sondern dienen als getrennte Stellglieder, die zur Ausmessung alternierend variiert werden, bis die Brückenspannung 
gleich null ist. 
Die Werte der bekannten Widerstände $R_2$, $R_3$ und $R_4$ und der Kapazität $C_4$ werden abgelesen und notiert. 
Diese Messung wird für zwei verschiedene Widerstände $R_2$ durchgeführt.

\subsection{Wien-Robinsion-Brücke}
\label{sec:durch-wien-robinson}
Bei der Messung mit der Wien-Robinsion-Brücke, wird nun die Frequenz $\nu$ in einem Intervall von $[\SI{100}{\hertz}, \SI{20000}{\hertz]}$ 
variiert. Dabei wird die Brückenspannung gemessen und notiert. Es werden mindestens $20$ Messdaten aufgenommen, wobei um das Minimum 
herum kleinere Messschritte gewählt werden. 