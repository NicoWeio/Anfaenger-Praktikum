\section{Auswertung}
  \subsection{Freie Schwingungen}
    Die Schwingungsdauern der frei schwingenden Pendel werden in folgender Tabelle wiedergegeben:
      \begin{table}
        \centering
          \caption{freie Schwingungsdauern T1 und T2.}
          \label{tab:aufgabe1}
          \sisetup{table-format=1.4}
          \begin{tabular}{S S S}
            \toprule
            {Pendellänge/m}  & {$T_{1}$} & {$T_{2}$} \\
            \midrule
            0.993            &   1.9400  &   1.9450 \\
            0.325            &   1.2274  &   1.2596 \\
            \bottomrule
          \end{tabular}
        \end{table}
    Die dem zugrundeliegenden Messdaten beziehen sich auf 5 Schwingungen. Es wurden zehn Messungen für jeweils Pendel 1
    und Pedel 2 durchgeführt, wobei die eingestellte Pendellänge von Pendel 1 $0.993$m und von Pendel 2 $0.995$m beträgt.
    Gemessen wurde die Pendellänge vom Gewichtsmittelpunkt bis zu der Auflagenadel des Pendels. Die  Schwingungsdauern
    wurden berechnet, indem der Mittelwert der 10 Messungen durch 5 geteilt wurde.
    Problematisch bei diesen Messungen ist, dass die Federhöhe in den Anweisungen nicht genauer spezifiziert ist.
    Solcherlei Ungenauigkeiten in der Anweisung werden im folgenden an den entsprechenden Punkten genannt und in der
    Diskussion abschließend behandelt. Bei obigen Messungen hing die Feder $0.288$m unter der Pendelaufhängung.
  \subsection{gleich- und gegensinnige Schwingungen}
    \begin{table}
      \centering
        \caption{gegensinnige Schwingungsdauer $T_{-}$ und gleichsinnige Schwingungsdauer $T_{+}$.}
        \label{tab:aufgabe23}
        \sisetup{table-format=1.4}
        \begin{tabular}{S S S S}
          \toprule
          {Pendellänge/m} & {$T_{+}$} & {$T_{-}$} & {$T_{-}$ Versuch 2}\\
          \midrule
          0.993           & 1.9772    & 1.9460   & 1.7408 \\
          0.325           & 1.0996   & 0.9594    &         \\
          \bottomrule
        \end{tabular}
    \end{table}
    Es wurden analog zu der Mesung der freien Schwingungsdauer 10 Messungen durchgeführt. Aus diesen wurde dann der Mittelwert
    gebildet und dieser durch 5 geteilt. Wenn man die Daten zu $T_{+}$ und $T_{-}$ bei einer Pendellänge von $0.993$m vergleicht,
    fällt auf dass zwischen der gleichsinnigen und der gegensinnigen Schwingung nur ein sehr geringer Unterschied herrscht. Dies
    liegt daran, dass die Feder sehr weit oben ($0.288$m unter der Aufhängung) an den Pendeln befestigt ist und daher nur bei
    sehr großen Auslenkungen eine Rolle spielt. Große Auslenkungen sind aber nicht realisierbar, da die Pendel sonst aneinander
    stoßen. Daher haben wir eine zweite Messreihe durchgeführt mit der Feder $0.786$m unterhalb der Aufhängung des Pendels. Die
    Daten dieser Messreihe sind in der Tabelle unter $T_{-}$ Versuch 2 zu finden. Der Einfluss der Federhöhe ist hier deutlich
    erkennbar.
  \subsection{Gekoppelte Schwingungen}
    \begin{table}
      \centering
        \caption{Schwingungsdauer $T$ und Schwebungsdauer $T_{S}$ einer gekoppelten Schwingung}
        \label{tab:aufgabe4}
        \sisetup{table-format=1.4}
        \begin{tabular}{S S S}
          \toprule
          {Pendellänge/m} & {$T$} & {$T_{S}$} \\
\label{sec:Auswertung}
