\section{Auswertung}
  \subsection{Messdaten}
    In den folgenden Tabellen sind die während des Experimentes aufgenommenen Daten aufgeführt:
    \begin{table}
      \centering
          \label{tab:freieschwingungen}
          \caption{Messwerte der freien Schwingungen A.}
          \sisetup{table-format=1.2}
          \begin{tabular}{S S}
            \toprule
            {$T_{1} \cdot 5/\si{\second}$} & {$T_{2} \cdot 5/\si{\second}$} \\
            \midrule
            9.75 & 9.72 \\
            9.69 & 9.91 \\
            9.53 & 9.69 \\
            9.65 & 9.56 \\
            9.65 & 9.60 \\
            9.91 & 9.85 \\
            9.60 & 9.66 \\
            9.78 & 9.50 \\
            9.78 & 9.97 \\
            9.66 & 9.79 \\
            \bottomrule
          \end{tabular}
        \end{table}
      \begin{table}
           \centering
              \label{tab:freieschwingungen}
              \caption{Messwerte der freien Schwingungen B.}
              \sisetup{table-format=1.2}
          \begin{tabular}{S S}
            \toprule
            {$T_{1} \cdot 5/\si{\second}$} & {$T_{2} \cdot 5/\si{\second}$} \\
            \midrule
            6.25 & 5.97 \\
            6.06 & 6.34 \\
            6.00 & 6.47 \\
            6.00 & 6.38 \\
            6.32 & 6.41 \\
            6.16 & 6.31 \\
            6.18 & 6.16 \\
            6.22 & 6.38 \\
            6.06 & 6.19 \\
            6.12 & 6.37 \\
            \bottomrule
          \end{tabular}
       \end{table}
    \begin{table}
      \centering
        \caption{Pendellängen und Federhöhen der verschiedenen Messreihen.}
        \label{tab:PendellängenFederhöhen}
        \sisetup{table-format=1.3}
        \begin{tabular}{S S S S}
          \toprule
          {Messreihe} & {Pendellänge Pendel 1/$\si{\meter}$} & {Pendellänge Pendel 2/$\si{\meter}$} & {Federhöhe/$\si{\meter}$} \\
          \midrule
          A & 0.993 & 0.995 & 0.786 \\
          B & 0.325 &       & 0.538 \\
          \bottomrule
        \end{tabular}
    \end{table}
    \begin{table}
      \centering
          \caption{Messreihe A der gleich- und gegensinnigen Schwingungen.}
          \label{tab:gleichgegensinnigeSchwingung}
          \sisetup{table-format=2.2}
          \begin{tabular}{S S S}
            \toprule
            {$T_{+} \cdot 5/\si{\second}$} & {$T_{-} \cdot 5/\si{\second}$} & {$T_{-} \cdot 5/\si{\second}$ 2} \\
            \midrule
             9.75 & 9.50 & 8.78 \\
             9.88 & 9.78 & 8.85 \\
             9.66 & 9.69 & 8.94 \\
             9.85 & 9.94 & 8.65 \\
            10.09 & 9.62 & 8.63 \\
            10.06 & 9.84 & 8.75 \\
             9.94 & 9.87 & 8.62 \\
             9.97 & 9.68 & 8.41 \\
             9.94 & 9.69 & 8.50 \\
             9.72 & 9.69 & 8.91 \\
            \bottomrule
          \end{tabular}
        \end{table}
    \begin{table}
      \centering
          \caption{Messreihe B der gleich- und gegensinnigen Schwingungen.}
          \label{tab:gleichgegensinnigeSchwingung}
          \sisetup{table-format=2.2}
          \begin{tabular}{S S}
            \toprule
            {$T_{+} \cdot 5/\si{\second}$} & {$T_{-} \cdot 5/\si{\second}$} \\
            \midrule
            5.69 & 4.88  \\
            5.72 & 5.10  \\
            5.34 & 4.75  \\
            5.32 & 4.62  \\
            5.78 & 4.94  \\
            5.47 & 4.60  \\
            5.53 & 4.75  \\
            5.41 & 4.84  \\
            5.32 & 4.78  \\
            5.40 & 4.71  \\
            \bottomrule
        \end{tabular}
      \end{table}
  \newpage
  \subsection{Freie Schwingungen}
    \label{sec:freieschwingungen}
    Die Schwingungsdauern der frei schwingenden Pendel werden in folgender Tabelle wiedergegeben:
      \begin{table}
        \centering
          \caption{freie Schwingungsdauern T1 und T2.}
          \label{tab:aufgabe1}
          \sisetup{table-format=1.4}
          \begin{tabular}{S S S S S}
            \toprule
            {Pendellänge/m}  & {$T_{1}/\si{\second}$} & {$\increment T_{1}/\si{\second}$} & {$T_{2}/\si{\second}$} & {$ \increment T_{2}/\si{\second}$} \\
            \midrule
            0.993            &   1.9400  &  0.020  &  1.9450  &  0.029 \\
            0.325            &   1.2274  &  0.021  &  1.2596  &  0.028\\
            \bottomrule
          \end{tabular}
        \end{table}
        \\
    Die dem zugrundeliegenden Messdaten beziehen sich auf 5 Schwingungen, daher wurden die Messwerte durch fünf geteilt.
    Der Mittelwert wurde unter Verwendung folgender Gleichung berechnet
      \begin{equation}
        \label{eqn:mittelwert}
        \bar{x}_\text{k} = \frac{1}{N} \sum_{k = 1}^{N} x_\text{k}.
      \end{equation}
    Von diesem Mittelwert wurde dann die Standardabweichung mit
      \begin{equation}
        \label{eqn:standardabweichung}
        \increment \bar{x} = \sqrt{\frac{1}{N(N-1)} \sum_{k = 1}^{N}(x_\text{k} - \bar{x})^2 }
      \end{equation}
      ermittelt. Dabei sind $x_\text{k}$ die Messwerte mit zufälliger Unsicherheit.
  \subsection{gleich- und gegensinnige Schwingungen}
    \label{sec:gleichgegensinnig}
    \begin{table}
      \centering
        \caption{gegensinnige Schwingungsdauer $T_{-}$ und gleichsinnige Schwingungsdauer $T_{+}$.}
        \label{tab:aufgabe23}
        \sisetup{table-format=1.4}
        \begin{tabular}{S S S S S S S}
          \toprule
          {Pendellänge/m} & {$T_{+}/\si{\second}$} & {$\increment T_{+}/\si{\second}$} & {$T_{-}/\si{\second}$} & {$\increment T_{-}/\si{\second}$} & {$T_{-}/\si{\second}$ 2} &
          {$\increment T_{-}/\si{\second}$ 2} \\
          \midrule
          0.993           & 1.9772  & 0.027  & 1.9460 & 0.025  & 1.7408 & 0.033 \\
          0.325           & 1.0996  & 0.033  & 0.9594 & 0.029  &        &       \\
          \bottomrule
        \end{tabular}
    \end{table}
    \noindent
    Es wurden analog zu der Messung der freien Schwingungsdauer 10 Messungen durchgeführt und ebenfalls analog zu \ref{sec:freieschwingungen} der Mittelwert sowie die
    Standardabweichung ermittelt. Auch hier mussten die Messwerte durch fünf geteilt werden aufgrund der Messung von fünf Schwingungen. Außerdem wurde eine zweite Messreihe
    der gegensinnigen Schwingungsdauer durchgeführt, auf die in der Diskussion näher eingegangen wird. Bei den zwei Messreihen ist dabei die Federhöhe gemäß Tabelle
    \ref{tab:PendellängenFederhöhen} variert worden.
  \subsection{Gekoppelte Schwingungen}
    Es wurden die beiden Pendel mit einer Feder gekoppelt und bei zwei verschiedenen Pendellängen $0.993$m und $0.325$m die Dauer
    von fünf Schwingungen sowie die Dauer einer Schwebung gemessen. Die Schwingungsdauer wurde dann bestimmt indem die Messwerte
    durch fünf geteilt wurden.
    \begin{table}
      \centering
        \caption{Schwingungsdauer $T$ und Schwebungsdauer $T_{S}$ einer gekoppelten Schwingung}
        \label{tab:aufgabe4}
        \sisetup{table-format=1.4}
        \begin{tabular}{S S S}
          \toprule
          {Pendellänge/m} & {$T/\si{\second}$} & {$T_{S}/\si{\second}$} \\
          \midrule
          0.993   &   1.4060 &     14.7300 \\
                  &   1.5060 &     14.9500 \\
                  &   1.4820 &     14.9500 \\
                  &   1.4060 &     14.1200 \\
                  &   1.4500 &     14.6900 \\
                  &   1.4320 &     14.6600 \\
                  &   1.4060 &     13.4000 \\
                  &   1.4180 &     14.8900 \\
                  &   1.4380 &     13.9200 \\
                  &   1.4320 &     13.9500 \\
          0.325   &   0.8860 &     8.1300 \\
                  &   1.1000 &     7.6900 \\
                  &   0.9520 &     8.0000 \\
                  &   0.8980 &     8.2800 \\
                  &   0.9160 &     7.4300 \\
                  &   0.9100 &     8.3500 \\
                  &   0.9200 &     8.2800 \\
                  &   0.8880 &     8.2500 \\
                  &   0.9440 &     7.2800 \\
                  &   0.9040 &     7.8100 \\
          \bottomrule
        \end{tabular}
      \end{table}
    \newpage
  \subsection{Der Kopplungsgrad der beiden Pendel}
  \label{sec:kopplungsgrad}
  Die beiden Kopplungsgrade wurden auf Basis der Messwerte für gleich- und gegensinnige Schwingungen (siehe \ref{tab:aufgabe23}) sowie Gleichung \eqref{eqn:kopplungskonstante}
  berechnet. Hier wurde für die Pendellänge von $0.993$m auch einmal die fehlerbehaftete Messung der
  gegensinnigen Schwingung verwendet, um zu demonstrieren, wie groß der Einfluss der Federhöhe sein kann. Damit ist auch die
  große Abweichung der beiden Kopplungsgrade zu erklären. Bei einer Pendellänge von $0.993$m hängt die Feder bei $0.786$m unter
  der Aufhängung des Pendels und bei einer Pendellänge von $0.325$m bei $0.538$m unter der Aufhängung.
    \begin{table}
      \centering
        \caption{Der Kopplungsgrad $K$ für zwei verschiedene Pendellängen}
          \label{tab:aufgabe5}
          \sisetup{table-format=1.4}
          \begin{tabular}{S S S S S}
            \toprule
            {Pendellänge/m} & {$K$}  & {$\increment K$} & {$K$ 2} & {$\increment K$ 2}\\
            \midrule
            0.993   &   0.0159 & 0.019 &  0.1267 & 0.023\\
            0.325   &   0.1356 & 0.04  &         &      \\
            \bottomrule
          \end{tabular}
        \end{table}
        \\
        Die Unsicherheit wurde hier nach der Gaußschen Fehlerfortpflanzung \eqref{eqn:fehlerfortpflanzung} berechnet.
        Bei $n$ Messgrößen $n_\text{1}$, $n_\text{2}$, ..., $n_\text{N}$ mit der Unsicherheit
        $\increment n_\text{1}$, $\increment n_\text{2}$, ..., $\increment n_\text{N}$ gilt für die Unsicherheit
        der abgeleiteten Größe
        \begin{equation}
          \label{eqn:fehlerfortpflanzung}
          \increment f = \sqrt{\left(\frac{\partial f}{\partial n_\text{1}} \right)^2 (\increment n_\text{1})^2 + \left(\frac{\partial f}{\partial n_\text{2}} \right)^2 (\increment
          n_\text{2})^2 + ... + \left(\frac{\partial
          f}{\partial  n_\text{n}} \right)^2 (\increment  n_\text{n})^2}.
        \end{equation}
    \subsection{Schwebungsdauer der gekoppelten Pendel}
      Aus den gegensinnigen und gleichsinnigen Schwingungen wurden folgende Schwebungsdauern berechnet:
      \begin{table}
        \centering
          \caption{Vergleich der gemessenen und der berechneten Schwebungsdauer $T_{S}$}
            \label{tab:aufgabe6}
            \sisetup{table-format=3.4}
            \begin{tabular}{S S S S S S S}
              \toprule
              {Pendellänge/m} & {$T_{S}/\si{\second}$} & {$\increment T_{S}/\si{\second}$} & {$T_{S}/\si{\second}$ V2} &
              {$\increment T_{S}/\si{\second}$ V2} & {$T_{S}/\si{\second}$ Messung} & {$ \increment T_{S}/\si{\second}$ Messung} \\
              \midrule
              0.993   &   123.3215 &  144.25 & 14.5597 & 2.7 & 14.4260 & 0.5\\
              0.325   &     7.5247 &    2.3  &  7.9500 & 0.4 &         &    \\
              \bottomrule
            \end{tabular}
          \end{table}
      \\
      Als Vergleichsgröße wird hier der Mittelwert der $10$ Messwerte aus Tabelle \ref{tab:aufgabe4} für die jeweilige
      Pendellänge verwendet.
      Die Unsicherheit wurde wieder analog zu Abschnitt \ref{sec:kopplungsgrad} mit der Gaußschen Fehlerfortpflanzung \eqref{eqn:fehlerfortpflanzung} berechnet.
      \subsection{Berechnung der Schwingungsfrequenzen}
        In folgender Tabelle werden die auf Basis der während des Versuches aufgenommenen Daten berechneten Schwingungsfrequenzen dargestellt:
        \begin{table}
          \centering
            \caption{berechnete Schwingungsfrequenzen.}
              \label{tab:aufgabe7}
              \sisetup{table-format=1.3}
              \begin{tabular}{S S S S S S}
                \toprule
                {$\omega_{+}$} & {$\increment \omega_{+}$} & {$\omega_{-}$} & {$\increment \omega_{-}$} & {$\omega_{S}$} & {$\increment \omega_{S}$} \\
                \midrule
                3.18 & 0.04 & 3.61 & 0.07 & 0.436 & 0.015 \\
                \bottomrule
              \end{tabular}
            \end{table}
            \\
      Die Werte für $\omega$ wurden mithilfe von Gleichung \eqref{eqn:omega} sowie $T_{+} = 1.9972 \pm 0.021$, \\ $T_{-} = 1.7408 \pm 0.033$ und $T_{S} = 14.4260 \pm 0.5$
      welche den Tabellen \ref{tab:aufgabe23} und \ref{tab:aufgabe6} zu entnehmen sind, berechnet. Die Messunsicherheit wurde mit der Gaußschen Fehlerfortpflanzung berechnet,
      siehe dazu Rechnung \eqref{eqn:fehlerfortpflanzung}.
\label{sec:Auswertung}
