\section{Diskussion}
\label{sec:Diskussion}
Im folgenden sind die in der Diskussion erwähnten Werte abgebildet:
\begin{table}
  \centering
    \caption{gegensinnige Schwingungsdauer $T_{-}$ und gleichsinnige Schwingungsdauer $T_{+}$.}
    \label{tab:diskussion1}
    \sisetup{table-format=1.4}
    \begin{tabular}{S S S S S S S}
      \toprule
      {Pendellänge/m} & {$T_{+}/\si{\second}$} & {$\increment T_{+}/\si{\second}$} & {$T_{-}/\si{\second}$} & {$\increment T_{-}/\si{\second}$} & {$T_{-}/\si{\second}$ 2} &
      {$\increment T_{-}/\si{\second}$ 2} \\
      \midrule
      0.993           & 1.9772  & 0.027  & 1.9460 & 0.025  & 1.7408 & 0.033 \\
      0.325           & 1.0996  & 0.033  & 0.9594 & 0.029  &        &       \\
      \bottomrule
    \end{tabular}
\end{table}
\begin{table}
  \centering
    \caption{Vergleich der gemessenen und der berechneten Schwebungsdauer $T_{S}$}
      \label{tab:diskussion2}
      \sisetup{table-format=3.4}
      \begin{tabular}{S S S S S S S}
        \toprule
        {Pendellänge/m} & {$T_{S}/\si{\second}$} & {$\increment T_{S}/\si{\second}$} & {$T_{S}/\si{\second}$ V2} &
        {$\increment T_{S}/\si{\second}$ V2} & {$T_{S}/\si{\second}$ Messung} & {$ \increment T_{S}/\si{\second}$ Messung} \\
        \midrule
        0.993   &   123.3215 &  144.25 & 14.5597 & 2.7 & 14.4260 & 0.5\\
        0.325   &     7.5247 &    2.3  &  7.9500 & 0.4 &         &    \\
        \bottomrule
      \end{tabular}
    \end{table}
    \begin{table}
      \centering
        \caption{Der Kopplungsgrad $K$ für zwei verschiedene Pendellängen}
          \label{tab:diskussion3}
          \sisetup{table-format=1.4}
          \begin{tabular}{S S S S S}
            \toprule
            {Pendellänge/m} & {$K$}  & {$\increment K$} & {$K$ 2} & {$\increment K$ 2}\\
            \midrule
            0.993   &   0.0159 & 0.019 &  0.1267 & 0.023\\
            0.325   &   0.1356 & 0.04  &         &      \\
            \bottomrule
          \end{tabular}
        \end{table}
        \\
%%hier Messdaten einfügen und diese bewerten!
Die Messung der Periodendauern kann bei diesem Versuch durch verschiedenste Faktoren beeinflusst werden, die zu Abweichungen
der gemessenen Werte zu den theoretischen Werten führen. Zudem sind die Gleichungen, aus welchen die theoretischen Werte
berechnet werden, vereinfacht, und weichen von einem realen Pendel ab.\\
Mögliche Faktoren, die die Messwerte beeinflussen können:
\begin{enumerate}
    \item \textit{Menschliche Fehler:}\\
        Da die Messungen nicht automatisiert durchgeführt wurden, sondern ein Mensch abschätzen musste, wann die maximale
        Amplitude des Pendels erreicht ist, entstehen bei den Messungen zusätzliche Fehler. Diese Fehler werden aber durch den in Kapitel
        \ref{sec:Durchführung} beschriebenen Messvorgang und die häufige Messung nur zu einem kleinen Fehler in dem Mittelwert
        führen. Ähnliches gilt für die Messung der Schwebungsdauer.
    \item \textit{Die Stoppuhr:}\\
        Bei der für den Versuch verwendeten Stoppuhr fiel auf, dass der Schalter für das Auslösen der Stoppuhr mit erhöhtem
        Kraftaufwand betätigt werden musste, damit das Messgerät reagierte. Dies könnte auch zu kleinen Verzögerungen bei der
        Messung geführt haben, die die Ergebnisse vielleicht sogar systematisch beeinflussen.
    \item \textit{Das Auslenken der Pendel:}\\
        Wie in Kapitel \ref{sec:Theorie} erläutert, sind gleich- und gegensinnige Schwingungen spezielle, gekoppelte Schwingungen bei
        denen für die Auslenkungen der Pendel $\alpha_1=\alpha_2$ bzw. $\alpha_1=\alpha_2$ gilt. Um diesen Zustand zu erreichen
        müssen die Pendel beide um den exakten Winkel ausgelenkt werden und auch zum selben Zeitpunkt los gelassen werden. Das
        präzise Auslenken der Pendel stellte sich jedoch als schwierig heraus, da weder eine Skala als Hilfestellung, noch
        eine andere technische Möglichkeit gegeben war, um beide Pendel präzise auszulenken. Bei dem gleichzeitigen Loslassen
        der Pendel fließt noch die Problematik mit ein, dass eine andere Person zeitgleich die Stoppuhr starten muss. Hier
        können sich demnach menschliche Fehler häufen.
\end{enumerate}
Mögliche mathematische Gründe, die Abweichungen erklären können:
\begin{enumerate}
    \item \textit{Die Kleinwinkelnäherung:}\\
        Wie in \ref{sec:Theorie} und \ref{sec:Durchführung} bereits beschrieben führt die Kleinwinkelnäherung \eqref{eqn:kleinwinkel}
        zu einer vereinfachten, aber auch mehr oder weniger stark fehlerbehafteten Rechnung. Gerade bei großen Auslenkungen ist die
        Periodendauer nicht mehr unabhängig von der Auslenkung. Auch die Federkonstante kann eine Abhängig von
        der Auslekung bekommen, falls diese zu groß wird. Da durch eine fehlende Skala für die Winkel nicht gut erkenntlich war,
        wie weit die Pendel ausgelenkt wurden, können auch hier Abweichungen auftreten.
    \item \textit{Das Trägheitsmoment:}\\
        Die Pendel besitzen als physikalische ein Trägheitsmoment, wie in Kapitel \ref{sec:Theorie} beschrieben. Dieses ist nach
        \eqref{eqn:traegheit} abhängig von der Masseverteilung der Pendel. Zwar wurde der Großteil der Masse durch den in
        Kapitel \ref{sec:Durchführung} beschriebenen Aufbau der Pendel vergleichsweise gut in einem kleinen Bereich zentriert,
        jedoch ist das Pendel eben ein physikalisches und kein mathematisches. Die Berechnung der Periodendauern mir dem
        Modell eines mathematischen Pendels fürht also zu weiteren Fehlern.
\end{enumerate}
Systematische Probleme des Experiments:
\begin{enumerate}
    \item \textit{Die Höhe der Feder:} \\
    In \ref{tab:diskussion1} wurde eine zweite Messreihe gestartet, mit dem Ziel, den Einfluss der Federhöhe näher zu bestimmen.
    Es fällt auf, dass der Unterschied zwischen der gleichsinnigen und der gegensinnigen Schwingung der ersten Messreihe nur sehr gering ist.
    Dies liegt daran, dass die Feder sehr weit oben (0.288m unter der Aufhängung) an den Pendeln befestigt ist und daher nur bei sehr großen
    Auslenkungen eine Rolle spielt. Große Auslenkungen sind jedoch aus mehreren Gründen nicht realisierbar, einerseits weil die Pendel sonst
    in der Mitte kollidieren und andererseits führt eine größere Auslenkung der Pendel nach \ref{sec:Theorie} auf Grund der Kleinwinkelnäherung
    \eqref{eqn:kleinwinkel} zu größeren Abweichungen von den Theoriewerten. Die zweite Messreihe wurde mit der
    Feder 0.786m unterhalb der Aufhängung des Pendels durchgeführt, was zu
    deutlich plausibleren Messwerten führte. Dies setzt sich auch in der Berechnung des Kopplungsgrades fort (siehe \ref{tab:diskussion3}),
    dort ergibt sich ein im Vergleich zu den anderen Werten ein sehr geringer Kopplungsgrad, was auf die kaum vorhandene Kopplung der beiden Pendel schließen lässt.
    In Tabelle \ref{tab:diskussion2} ist dies ebenfalls zu sehen.
\end{enumerate}
Die vorliegenden Fehlerquellen könnten minimiert werden, indem die Messapparatur so verbessert wird, dass die Auslenkungen der Pendel
und die Zeitmessung automatisiert vorgenommen werden. Des weiteren ist ein komplexeres mathematisches Modell wahrscheinlich geeigneter,
um die theoretischen Werte zu berechnen. Ebenfalls sollte die Federhöhe in der Durchführungsanweisung spezifiziert werden.
