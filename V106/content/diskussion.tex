\section{Diskussion}
  Die Genauigkeit der Messungen könnten durch verschiedene Maßnahmen verbessert werden. Worauf im obigen Verlauf der
  Durchführung bereits hingewiesen wurde, die Wichtigkeit der Federhöhe, 
\label{sec:Diskussion}

Die Messung der Periodendauern kann bei diesem Versuch durch verschiedenste Faktoren beeinflusst werden, die zu Abweichungen
der gemessenen Werte zu den theoretischen Werten führen. Zudem sind die Gleichungen, aus welchen die theoretischen Werte 
berehnet werden vereinfacht und weichen von einem realen Pendel ab.\\ 
Mögliche Faktoren, die die Messwerte beeinflussen können:
\begin{enumerate}
    \item \textit{Menschliche Fehler:}\\
        Da die Messungen nicht automatisiert durchgeführt wurden, sondern ein Mensch abschätzen musste, wann die maximale 
        Amplitude des Pendels erreicht ist, entstehen bei den Messungen zusätzliche Fehler. Diese Fehler werden aber durch den in Kapitel
        \ref{sec:Durchführung} beschriebenen Messvorgang und die häufige Messung nur zu einem kleinen Fehler in dem Mittelwert
        führen.
    \item \textit{Die Stoppuhr:}\\
        Bei der für den Versuch verwendeten Stoppuhr fiel auf, dass der Schalter für das Auslösen der Stoppuhr mit erhöhtem 
        Kraftaufwand betätigt werden musste, damit das Messgerät reagierte. Dies könnte auch zu kleinen Verzögerungen bei der 
        Messung geführt haben, die die Ergebnisse vielleicht sogar systematisch beeinflussen.
    \item \textit{Das Auslenken der Pendel:}\\
        Wie in Kapitel \ref{sec:Theorie} erläutert, sind gleich- und gegensinnige Schwingungen spezielle, gekoppelte Schwingungen bei 
        denen für die Auslenkungen der Pendel $\alpha_1=\alpha_2$ bzw. $\alpha_1=\alpha_2$ gilt. Um diesen Zustand zu erreichen
        müssen die Pendel beide um den exakten Winkel ausgelenkt werden und auch zum selben Zeitpunkt los gelassen werden. Das 
        präzise Auslenken der Pendel stellte sich jedoch als schwierig heraus, da weder eine Skala als Hilfestellung, noch 
        eine andere technische Möglichkeit gegeben war, um beide Pendel präzise auszulenken. Bei dem gleichzeitigen Loslassen
        der Pendel fließt noch die Problematik mit ein, dass eine andere Person zeitgleich die Stoppuhr starten muss. Hier 
        können sich demnach menschliche Fehler häufen.
\end{enumerate}
Mögliche mathematische Gründe, die Abweichungen erklären können:
\begin{enumerate}
    \item \textit{Die Kleinwinkelnäherung:}\\
        Wie in \ref{sec:Theorie} und \ref{sec:Durchführung} bereits beschrieben führt die Kleinwinkelnäherung \eqref{eqn:kleinwinkel}
        zu einer vereinfachten, aber auch mehr oder weniger stark fehlerbehafteten Rechnung. Gerade bei großen Auslenkungen ist die
        Periodendauer nicht mehr unabhängig von der Auslenkung. Auch die Federkonstante kann eine Abhängig von
        der Auslekung bekommen, falls diese zu groß wird. Da durch eine fehlende Skala für die Winkel nicht gut erkenntlich war,
        wie weit die Pendel ausgelenkt wurden, können auch hier Abweichungen auftreten.
    \item \textit{Das Trägheitsmoment:}\\
        Die Pendel besitzen als physikalische ein Trägheitsmoment, wie in Kapitel \ref{sec:Theorie} beschrieben. Dieses ist nach 
        \eqref{eqn:traegheit} abhängig von der Masseverteilung der Pendel. Zwar wurde der Großteil der Masse durch den in 
        Kapitel \ref{sec:Durchführung} beschriebenen Aufbau der Pendel vergleichsweise gut in einem kleinen Bereich zentriert,
        jedoch ist das Pendel eben ein physikalisches und kein mathematisches. Die Berechnung der Periodendauern mir dem 
        Modell eines mathematischen Pendels fürht also zu weiteren Fehlern.  
\end{enumerate}
Die vorliegenden Fehlerquellen könnten minimiert werden, idem man die Messaperatur so verbessert, dass die Auslenkungen der Pendel
und die Zeitmessung automatisiert vorgenommen werden. Desweiteren ist ein komplexeres mathematisches Modell wahrscheinlich geeigneter,
um die theoretischen Werte zu berechnen.