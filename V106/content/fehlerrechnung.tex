\section{Fehlerrechnung}
\label{sec:Fehlerrechnung}
\subsection{Berechnung des Mittelwerts}
  \label{sec:mittelwert}
  Für den Mittelwert gilt
  \begin{equation}
    \bar{x}_\text{k} = \frac{1}{N} \sum_{k = 1}^{N} x_\text{k}.
  \end{equation}
\subsection{Berechnung der Standardabweichung}
  \label{sec:standardabweichung}
  Die Standardabweichung berechnet sich wie folgt:
  \begin{equation}
    \increment \bar{x} = \sqrt{\frac{1}{N(N-1)} \sum_{k = 1}^{N}(x_\text{k} - \bar{x})^2 }
  \end{equation}
  mit Messwerten $x_\text{k}$ mit zufälligem Fehler.
\subsection{Fehlerfortpflanzung nach Gauß}
  \label{sec:fehlerfortpflanzung}
  Für $n$ Messgrößen $n_\text{1}$, $n_\text{2}$, ..., $n_\text{N}$ mit der Unsicherheit
  $\increment n_\text{1}$, $\increment n_\text{2}$, ..., $\increment n_\text{N}$. Für die Unsicherheit
  der abgeleiteten Größe gilt
  \begin{equation}
    \increment f = \sqrt{\left(\frac{\partial f}{\partial n_\text{1}} \right)^2 (\increment n_\text{1})^2 + \left(\frac{\partial f}{\partial n_\text{2}} \right)^2 (\increment
    n_\text{2})^2 + ... + \left(\frac{\partial
    f}{\partial  n_\text{n}} \right)^2 (\increment  n_\text{n})^2}
  \end{equation}
\subsection{Lineare Regression}
  \label{sec:regression}
  Für ($x_\text{1}$, $y_\text{1} \pm \sigma$), ..., ($x_\text{N}$, $y_\text{N} \pm \sigma$) linear abhängige Größen und der Geradengleichung $y = m \cdot x + b$ ergibt sich die
  Regression zu
  \begin{align}
    \hat{m}  & = \frac{\bar{xy} - \bar{x} \cdot \bar{y}}{\bar{x^2} - \bar{x}^2} \\
    \hat{b}  & = \bar{y} - \hat{m} \bar{x}
  \end{align}
  Für die Unsicherheit $\sigma_\text{m}^2$ und $\sigma_\text{b}^2$ gilt dann
  \begin{align}
    \sigma_\text{m}^2 & = \frac{\sigma^2}{N(\bar{x^2} - \bar{x}^2)} \\
    \sigma_\text{b}^2 & = \frac{\sigma^2 \bar{x^2}}{N(\bar{x^2} - \bar{x}^2)}
  \end{align}
