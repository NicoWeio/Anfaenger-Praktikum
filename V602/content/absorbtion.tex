\clearpage
\subsubsection*{Zink}
\begin{figure}[H]
    \centering
    \includegraphics[width=\textwidth]{build/plot_zink.pdf}
    \caption{Absorptionsspektrum von Zink.}
    \label{fig:zink}
\end{figure}
\begin{table}[H]                                                                                   
    \centering                                                                                     
        \caption{Wertepaare für die Extrema und den berechneten Mittelpunkt für Zink.}                      
        \label{tab:Zn}                                                                        
        \sisetup{table-format=3.0}                                                                 
        \begin{tabular}{S S[table-format=2.2] S}                                                   
          \toprule                                                                                 
          {Punkt} & {$\theta [\si{\degree}]$} & {$N$}\\                                            
          \midrule                                                                                 
          {$I_{K,Min  }$} & 18.3  &54   \\
          {$I_{K,Max  }$} & 19.0  &102  \\
          {$I_{K,Mitte}$} & 18.67 &78   \\
          \bottomrule                                                                              
        \end{tabular}                                                                              
      \end{table}                                                                                  
Daraus ergibt sich                                                                                 
\begin{align*}                                                                                     
    E_\text{Zink} &= \SI{9.62}{\kilo\electronvolt}\\                  
    \sigma_{K, \text{Zink}} &= \num{3.62}                      
\end{align*}                                                                                       

\clearpage
\subsubsection*{Gallium}
\begin{figure}[H]
    \centering
    \includegraphics[width=\textwidth]{build/plot_gallium.pdf}
    \caption{Absorptionsspektrum von Gallium.}
    \label{fig:gallium}
\end{figure}
\begin{table}[H]                                                                                   
    \centering                                                                                     
        \caption{Wertepaare für die Extrema und den berechneten Mittelpunkt für Gallium.}                      
        \label{tab:Ga}                                                                        
        \sisetup{table-format=3.0}                                                                 
        \begin{tabular}{S S[table-format=2.2] S}                                                   
          \toprule                                                                                 
          {Punkt} & {$\theta [\si{\degree}]$} & {$N$}\\                                            
          \midrule                                                                                 
          {$I_{K,Min  }$} &17.1&66\\
          {$I_{K,Max  }$} &17.9&122\\
          {$I_{K,Mitte}$} &17.34&94\\
          \bottomrule                                                                              
        \end{tabular}                                                                              
      \end{table}                                                                                  
Daraus ergibt sich                                                                                 
\begin{align*}                                                                                     
    E_\text{Gallium} &= \SI{10.33}{\kilo\electronvolt}\\                  
    \sigma_{K, \text{Gallium}} &= \num{3.68}                      
\end{align*}                                                                                       

\clearpage
\subsubsection*{Brom}
\begin{figure}[H]
    \centering
    \includegraphics[width=\textwidth]{build/plot_brom.pdf}
    \caption{Absorptionsspektrum von Brom.}
    \label{fig:brom}
\end{figure}
\begin{table}[H]                                                                                   
    \centering                                                                                     
        \caption{Wertepaare für die Extrema und den berechneten Mittelpunkt für Brom.}                      
        \label{tab:Br}                                                                        
        \sisetup{table-format=3.0}                                                                 
        \begin{tabular}{S S[table-format=2.2] S}                                                   
          \toprule                                                                                 
          {Punkt} & {$\theta [\si{\degree}]$} & {$N$}\\                                            
          \midrule                                                                                 
          {$I_{K,Min  }$} &13.0&9\\
          {$I_{K,Max  }$} &13.6&27\\
          {$I_{K,Mitte}$} &13.20&18\\
          \bottomrule                                                                              
        \end{tabular}                                                                              
      \end{table}                                                                                  
Daraus ergibt sich                                                                                 
\begin{align*}                                                                                     
    E_\text{Brom} &= \SI{13.48}{\kilo\electronvolt}\\                  
    \sigma_{K, \text{Brom}} &= \num{3.84}                      
\end{align*}                                                                                       

\clearpage
\subsubsection*{Rubidium}
    \begin{figure}[H]
    \centering
    \includegraphics[width=\textwidth]{build/plot_rubidium.pdf}
    \caption{Absorptionsspektrum von Rubidium.}
\label{fig:rubidium}
\end{figure}
\begin{table}[H]                                                                                   
    \centering                                                                                     
        \caption{Wertepaare für die Extrema und den berechneten Mittelpunkt für Rubidium.}                      
        \label{tab:Rb}                                                                        
        \sisetup{table-format=3.0}                                                                 
        \begin{tabular}{S S[table-format=2.2] S}                                                   
          \toprule                                                                                 
          {Punkt} & {$\theta [\si{\degree}]$} & {$N$}\\                                            
          \midrule                                                                                 
          {$I_{K,Min  }$} &11.4&10\\
          {$I_{K,Max  }$} &12.1&64\\
          {$I_{K,Mitte}$} &11.77&37\\
          \bottomrule                                                                              
        \end{tabular}                                                                              
      \end{table}                                                                                  
Daraus ergibt sich                                                                                 
\begin{align*}                                                                                     
    E_\text{Rubidium} &= \SI{15.09}{\kilo\electronvolt}\\                  
    \sigma_{K, \text{Rubidium}} &= \num{4.08}                      
\end{align*}                                                                                       

\clearpage
\subsubsection*{Strontium}
\begin{figure}[H]
    \centering
    \includegraphics[width=\textwidth]{build/plot_strontium.pdf}
    \caption{Absorptionsspektrum von Strontium.}
    \label{fig:strontium}
\end{figure}
\begin{table}[H]                                                                                   
    \centering                                                                                     
        \caption{Wertepaare für die Extrema und den berechneten Mittelpunkt für Strontium.}                      
        \label{tab:Sr}                                                                        
        \sisetup{table-format=3.0}                                                                 
        \begin{tabular}{S S[table-format=2.2] S}                                                   
          \toprule                                                                                 
          {Punkt} & {$\theta [\si{\degree}]$} & {$N$}\\                                            
          \midrule                                                                                 
          {$I_{K,Min  }$} &10.7&40\\
          {$I_{K,Max  }$} &11.6&196\\
          {$I_{K,Mitte}$} &11.09&118\\
          \bottomrule                                                                              
        \end{tabular}                                                                              
      \end{table}                                                                                  
Daraus ergibt sich                                                                                 
\begin{align*}                                                                                     
    E_\text{Strontium} &= \SI{16.00}{\kilo\electronvolt}\\                  
    \sigma_{K, \text{Strontium}} &= \num{4.12}                      
\end{align*}                                                                                       

\clearpage
\subsubsection*{Zirkonium}
\begin{figure}[H]
    \centering
    \includegraphics[width=\textwidth]{build/plot_zirkonium.pdf}
    \caption{Absorptionsspektrum von Zirkonium.}
    \label{fig:zirkonium}
\end{figure}
\begin{table}[H]                                                                                   
    \centering                                                                                     
        \caption{Wertepaare für die Extrema und den berechneten Mittelpunkt für Zirkonium.}                      
        \label{tab:Zr}                                                                        
        \sisetup{table-format=3.0}                                                                 
        \begin{tabular}{S S[table-format=2.2] S}                                                   
          \toprule                                                                                 
          {Punkt} & {$\theta [\si{\degree}]$} & {$N$}\\                                            
          \midrule                                                                                 
          {$I_{K,Min  }$} &9.5&112\\
          {$I_{K,Max  }$} &10.4&301\\
          {$I_{K,Mitte}$} &9.96&206.5\\
          \bottomrule                                                                              
        \end{tabular}                                                                              
      \end{table}                                                                                  
Daraus ergibt sich                                                                                 
\begin{align*}                                                                                     
    E_\text{Zirkonium} &= \SI{17.80}{\kilo\electronvolt}\\                  
    \sigma_{K, \text{Zirkonium}} &= \num{4.31}                      
\end{align*}                                                   