\newpage
\section{Diskussion}
\label{sec:Diskussion}
Die Ergebnisse aus Kapitel \ref{sec:Auswertung} sind in der folgenden Tabelle \ref{tab:diss1} noch einmal zusammengefasst.
\begin{table}[H]
    \centering
        \caption{Zusammenfassung der Ergebnisse und Vergleich mit Literaturwerten.}
        \label{tab:diss1}
        \sisetup{table-format=1.4}
        \begin{tabular}{S S S S}
          \toprule
          {Größe} & {Wert} & {Literatur} & {p [\%]}\\
          \midrule
          {$\theta_{glanz}$} & {$\SI{28.2 }{\degree}           $}    & {$\SI{28.0 }{\degree}           $} & 0.71  \\
          {$K_{\alpha}$}     & {$\SI{8.044}{\kilo\electronvolt}$}    & {$\SI{8.048}{\kilo\electronvolt}$} & 0.047 \\
          {$K_{\beta} $}     & {$\SI{8.915}{\kilo\electronvolt}$}    & {$\SI{8.907}{\kilo\electronvolt}$} & 0.091 \\
          {$\sigma_{Zn}$}    & 3.62                                  & 3.57                               & 1.32  \\
          {$\sigma_{Ga}$}    & 3.68                                  & 3.62                               & 1.64  \\
          {$\sigma_{Br}$}    & 3.84                                  & 3.85                               & 0.29  \\
          {$\sigma_{Rb}$}    & 4.08                                  & 3.95                               & 3.16  \\
          {$\sigma_{Sr}$}    & 4.12                                  & 4.01                               & 2.78  \\
          {$\sigma_{Zr}$}    & 4.31                                  & 4.11                               & 4.79  \\
          {$R_y$}            & {$\SI{12.52\pm 0.13}{\electronvolt}$} & {$\SI{13.61}{\electronvolt}$}      & 7.97  \\
          {$R$}              & {$\SI{3.03\pm 0.03}{\peta\hertz}$}    & {$\SI{3.29}{\peta\hertz}$}         & 7.97  \\
          \bottomrule
        \end{tabular}
      \end{table}
Die prozentuale Abweichung $p$ ist dabei durch 
\begin{equation*}
  p=\frac{ideal-mess}{ideal}\cdot \num{100}
\end{equation*}
gegeben.
Bei der Überprüfung der Bragg-Bedingung liegt die in Kapitel \ref{sec:brag} berechnete Abweichung bei $\Delta\theta_{glanz}=\SI{0.2}{\degree}$.
Diese Abweichung liegt deutlich unter einem Grad und somit in der vorgeschriebenen Toleranz. Wäre die Abweichung der Position des Maximums 
größer als $\SI{3}{\degree}$, wäre ein systematischer Messfehler anzunehmen. Somit wären dann auch alle Folgenden Messungen nicht aussagekräftig.
\\\noindent
Wie in Tabelle \ref{tab:diss1} dargestellt, gleichen auch die berechneten Photonenenergien bei $K_\alpha$ und $K_\beta$
den Literaturwerten in sehr hohem Maße. Die Abweichung liegt unter $\SI{0.1}{\percent}$. Dies lässt auf eine sehr genaue Messung schließen.
\\\noindent
Die minimale Wellenlänge \eqref{eqn:lammin} kann nicht aus den Messwerten abgelesen werden, da die Werte dafür nicht in dem Messintervall von 
$\SI{8.0}{\degree}\leq\theta\leq\SI{25}{\degree}$ liegen. Durch Einsetzen der minimalen Wellenlänge \eqref{eqn:lammin} in die Braggbedingung
\eqref{eqn:braggtheorie} ergibt sich für den Winkel $\theta_{\lambda_\text{min}}$ 
\begin{equation*}
  \theta_{\lambda_\text{min}}=\arcsin{\left(\frac{nhc}{2de_0U}\right)}\approx\SI{5.05}{\degree} \;.
\end{equation*} 
\\\noindent
Auch die Abschirmkonstanten entsprechen weitestgehend den Literaturwerten \cite{AP05}, sodass die Abweichungen zwischen $\SI{4.79}{\percent}$
und $\SI{0.29}{\percent}$ liegen. Die Ergebnisse wären wahrscheinlich deutlich genauer, wenn mehr Messdaten aufgenommen worden wären, da
dann die Bestimmung der K-Kante präziser erfolgt wäre. Auffällig ist, dass die Abweichung für höhere Ordungszahlen steigt. Dieser Umstand 
könnte darin begründet sein, dass Gleichung \ref{eqn:abschirmktheorie} die Feinstruktur vernachlässigt. Diese gewinnt bei steigender Ordungszahl
jedoch an Relevanz, sodass der Fehler immer größer wird.
\\\noindent
Bei der Rydbergenergie und der Rydbergfrequenz sind die prozentualen Abweichungen von den Literaturwerten \cite{AP06} nicht so gering, wie bei
den restlichen Ergebnissen. Durch die Berechnung mittels Regressionsgerade führen die Abweichungen der Absorbtionsenergie wahrscheinlich
zu größeren Abweichungen der Parameter.


%warum kann die minimale Wellenlänge / maximale Energie aus dem Emissionsspektrum nicht abgelesen werden?
% 