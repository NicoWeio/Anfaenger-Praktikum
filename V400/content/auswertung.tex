\section{Auswertung}
\label{sec:Auswertung}
In den folgenden Tabellen sind die gemessenen Werte zusammenfassend dargestellt.

\begin{table}
  \centering
  \caption{Die Messwerte der Reflexion an einem Spiegel für verschiedene Winkel.}
  \label{tab:MessungAufgabe1}
  \sisetup{table-format=2.1}
  \begin{tabular}{S S}
    \toprule
    {Einfallswinkel/\si{\degree}} & {Ausfallswinkel\si{\degree}} \\
    \midrule
    69   &  69   \\
    42   &  43   \\
    30.5 & 30.5  \\
    55   & 56    \\
    72   & 72    \\
    20   & 20    \\
    35   & 35    \\
    52   & 52.5  \\
    \bottomrule
  \end{tabular}
\end{table}

\begin{table}
  \centering
  \caption{Die Messwerte der Brechung an einer planparallelen Platte der Messung 1 für verschiedene Winkel.}
  \label{tab:MessungAufgabe2}
  \sisetup{table-format=2.2}
  \begin{tabular}{S S}
    \toprule
    {Einfallswinkel/\si{\degree}} & {Brechungswinkel/\si{\degree}} \\
    \midrule
    69 & 39    \\
     0 &  0    \\
    10 &  7    \\
    30 & 20    \\
    40 & 25.5  \\
    75 & 40.5  \\
    55 & 33.5  \\
    38 & 24.25 \\
    \bottomrule
  \end{tabular}
\end{table}

\begin{table}
  \centering
  \caption{Die Messwerte der Brechung an einer planparallelen Platte der Messung 2 für verschiedene Winkel.}
  \label{tab:MessungAufgabe3}
  \sisetup{table-format=2.1}
  \begin{tabular}{S S}
    \toprule
    {Einfallswinkel/\si{\degree}} & {Brechungswinkel/\si{\degree}} \\
    \midrule
    0  &  0     \\
   45  & 28.5   \\
   60  & 36     \\
   20  & 13.5   \\
   51  & 31.5   \\
    3  &  3.5   \\
    \bottomrule
  \end{tabular}
\end{table}

\begin{table}
  \centering
  \caption{Die Messwerte der Brechung an einem Prsima für verschiedene Winkel.}
  \label{tab:MessungAufgabe4}
  \sisetup{table-format=2.1}
  \begin{tabular}{S S S}
    \toprule
    {Einfallswinkel/\si{\degree}} & {Ausfallswinkel $L_{Gruen}$/\si{\degree}} & {Ausfallswinkel $L_{Rot}$/\si{\degree}} \\
    \midrule
    30  & 82   & 84   \\
    35  & 73.5 & 74.5 \\
    39  & 60   & 60.7 \\
    46  & 48   & 49.5 \\
    60  & 42   & 43   \\
    \bottomrule
  \end{tabular}
\end{table}

\begin{table}
  \centering
  \caption{Die gemessenen Beugungsmaxima der Brechung an einem Strichgitter mit 600 Linien/mm.}
  \label{tab:MessungAufgabe51}
  \sisetup{table-format=2.1}
  \begin{tabular}{S S S S}
    \toprule
    {Laserposition/\si{\degree}} & $B_{max1}/\si{\degree}$ & $B_{max2}/\si{\degree}$ & $B_{max3}/\si{\degree}$ \\
    \midrule
    0  & 22.5 & 0 & 22.5 \\
    \bottomrule
  \end{tabular}
\end{table}

\begin{table}
  \centering
  \caption{Die gemessenen Beugungsmaxima der Brechung an einem Strichgitter mit 300 Linien/mm.}
  \label{tab:MessungAufgabe52}
  \sisetup{table-format=2.1}
  \begin{tabular}{S S S S S S}
    \toprule
    {Laserposition/\si{\degree}} & $B_{max1}/\si{\degree}$ & $B_{max2}/\si{\degree}$ & $B_{max3}/\si{\degree}$ & $B_{max4}/\si{\degree}$ & $B_{max5}/\si{\degree}$ \\
    \midrule
    0 & 22 & 10.7 & 0 & 10.7 & 22 \\
    \bottomrule
  \end{tabular}
\end{table}

\begin{table}
  \centering
  \caption{Die gemessenen Beugungsmaxima der Brechung an einem Strichgitter mit 100 Linien/mm. Der Laser steht wieder bei 0 \si{\degree}.}
  \label{tab:MessungAufgabe53}
  \sisetup{table-format=2.1}
  \begin{tabular}{S S}
    \toprule
    {Maximum} & {Position/\si{\degree}} \\
    \midrule
    1   & 26 \\
    2   & 22 \\
    3   & 18 \\
    4  & 14 \\
    5   & 11 \\
    6   & 7.2 \\
    7   & 3.7 \\
    8  &   0  \\
    9  &   3.7\\
    10  &   7.2\\
    11   & 11 \\
    12   & 14 \\
    13   & 18 \\
    14   & 22 \\
    15   & 26 \\
    \bottomrule
  \end{tabular}
\end{table}

\subsection{Reflexion}
\label{sec:reflexionauswertung}
\subsection{Brechung}
\label{sec:brechungauswertung}
\subsection{Planparalle Platten}
\label{sec:planplatteauswertung}
\subsection{Prisma}
\label{sec:prismaauswertung}
\subsection{Beugung am Gitter}
\label{sec:beugungauswertung}
