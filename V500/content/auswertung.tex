\section{Auswertung}
\label{sec:Auswertung}
Die gemessenen Werte sind in folgenden Tabellen dargestellt.
\begin{table}[H]
  \centering
  \caption{Messdaten des Versuches.}
  \label{tab:messdaten1}
  \sisetup{table-format=3.2}
  \begin{tabular}
    {S S S[table-format=2.3] S[table-format=1.3] }
    \toprule
    {$U_g$} & {$I_{gruen}$} & {$I_{blaugruen}$} & {$I_{blau}$} \\
    \midrule
    2.0  &  2.1   &   0.15    &   0.8   \\
    1.8  &  2.0   &   0.135   &   0.7   \\
    1.6  &  1.8   &   0.125   &   0.6   \\
    1.4  &  1.5   &   0.11    &   0.5   \\
    1.2  &  1.2   &   0.1     &   0.52  \\
    1.0  &  1.0   &   0.085   &   0.44  \\
    0.8  &  0.8   &   0.073   &   0.36  \\
    0.6  &  0.6   &   0.06    &   0.28  \\
    0.4  &  0.35  &   0.043   &   0.2   \\
    0.2  &  0.24  &   0.028   &   0.16  \\
    0.01 &  0.1   &   0.015   &   0.105 \\
   -0.01 &  0.5   &   0.018   &   0.11  \\
   -0.2  &  0.2   &   0.01    &   0.045 \\
   -0.4  &  0.03  &   0.004   &   0.042 \\
   -0.6  &  0     &   0.001   &   0.024 \\
   -0.8  &  0     &   0       &   0.008 \\
   -1.0  &        &   0       &   0.002 \\
   -1.2  &        &           &   0     \\
   -1.4  &        &           &   0     \\
   \bottomrule
 \end{tabular}
\end{table}
\noindent
\begin{table}[H]
  \centering
  \caption{Messdaten des Versuches.}
  \label{tab:messdaten2}
  \sisetup{table-format=3.2}
  \begin{tabular}
    {S S}
    \toprule
    {$U_g$} & {$I_{gelb}$} \\
    \midrule
    19    &  2.3  \\
    18    &  2.3  \\
    17    &  2.25 \\
    16    &  2.2  \\
    15    &  2.2  \\
    14    &  2.15 \\
    13    &  2.05 \\
    12    &  2.05 \\
    11    &  2.00 \\
    10    &  2.00 \\
     9    &  1.95 \\
     8    &  1.85 \\
     7    &  1.8  \\
     6    &  1.7  \\
     5    &  1.55 \\
     4    &  1.4  \\
     3    &  1.2  \\
     2    &  0.87 \\
     1.8  &  0.87 \\
     1.6  &  0.81 \\
     1.4  &  0.74 \\
     1.2  &  0.68 \\
     1.0  &  0.60 \\
     0.8  &  0.53 \\
     0.6  &  0.43 \\
     0.4  &  0.32 \\
     0.2  &  0.20 \\
     0.01 &  0.10 \\
    -0.01 &  0    \\
    -0.2  &  0    \\
    \bottomrule
  \end{tabular}
 \end{table}
\noindent
Es wird ein Dunkelstrom von $I_{Dunkel} = 0.03 \si{\nano\ampere}$ gemessen.
Dieser wird vor weiteren Rechnungen von allen gemessenen Strömen abgezogen.
\noindent
\subsection{Bestimmung der Grenzspannung}
Für die grüne Spektrallinie mit einer Wellenlänge von $\lambda = 546 \si{\nano\meter}$
ist in folgender Abbildung die Wurzel des gemessenen Photostroms gegen die Bremsspannung
aufgetragen.
\begin{figure}[H]
  \centering
  \includegraphics[scale=0.8]{build/plotgrün.pdf}
  \label{fig:plotgrün}
  \caption{Die Messwerte der grünen Spektrallinie sowie ein linearer Fit.}
\end{figure}
\noindent
An die Messwerte ist eine lineare Funktion der Form $mx +b$ gefittet. Es ergeben
sich die Parameter
\begin{align*}
  m = & 0.543 \pm 0.029 \\
  b = & 0.456 \pm 0.029. \\
\end{align*}
Die Grenzspannung $U_g$ ergibt sich hier zu $ U_g = -0.84 \pm 0.07$ \si{\nano\ampere}. \\
\noindent
Für die blaugrüne Spektrallinie mit einer Wellenlänge von $\lambda = 492 \si{\nano\meter}$
ist in folgender Abbildung die Wurzel des gemessenen Photostroms gegen die Bremsspannung
aufgetragen.
\begin{figure}[H]
  \centering
  \includegraphics[scale=0.8]{build/plotblaugrün.pdf}
  \label{fig:plotblaugrün}
  \caption{Die Messwerte der blaugrünen Spektrallinie sowie ein linearer Fit.}
\end{figure}
\noindent
An die Messwerte ist eine lineare Funktion der Form $mx +b$ gefittet. Es ergeben
sich die Parameter
\begin{align*}
  m = & 0.141 \pm 0.005 \\
  b = & 0.137 \pm 0.005.\\
\end{align*}
Die Grenzspannung $U_g$ ergibt sich hier zu $ U_g = -0.97 \pm 0.05$ \si{\nano\ampere}. \\
\noindent
Für die blaue Spektrallinie mit einer Wellenlänge von $\lambda = 435 \si{\nano\meter}$ \cite{AP02}
ist in folgender Abbildung die Wurzel des gemessenen Photostroms gegen die Bremsspannung
aufgetragen.
\begin{figure}[H]
  \centering
  \includegraphics[scale=0.8]{build/plotblau.pdf}
  \label{fig:plotblau}
  \caption{Die Messwerte der blauen Spektrallinie sowie ein linearer Fit.}
\end{figure}
\noindent
An die Messwerte ist eine lineare Funktion der Form $mx +b$ gefittet. Es ergeben
sich die Parameter
\begin{align*}
  m = & 0.284 \pm 0.007 \\
  b = & 0.346 \pm 0.007.\\
\end{align*}
Die Grenzspannung $U_g$ ergibt sich hier zu $ U_g = -1.22 \pm 0.04$ \si{\nano\ampere}. \\
\noindent
Für die gelbe Spektrallinie mit einer Wellenlänge von $\lambda = 577 \si{\nano\meter}$
ist in folgender Abbildung die Wurzel des gemessenen Photostroms gegen die Bremsspannung
aufgetragen.
\begin{figure}[H]
  \centering
  \includegraphics[scale=0.8]{build/plotgelb.pdf}
  \label{fig:plotgelb}
  \caption{Die Messwerte der gelben Spektrallinie sowie ein linearer Fit.}
\end{figure}
\noindent
An den Teil der Messwerte, welcher näherungsweise linear verläuft, ist eine lineare Funktion der Form $mx +b$ gefittet. Es ergeben sich die Parameter
\begin{align*}
  m = & 0.0138 \pm 0.0008 \\
  b = & 1.266 \pm 0.011.\\
\end{align*}
Die Grenzspannung $U_g$ ergibt sich hier zu $ U_g = -92 \pm 5$ \si{\nano\ampere}. \\
\subsection{Bestimmung der Austrittsarbeit sowie dem Verhältnis $h/e^0$}
Das Verhältnis $\frac{h}{e_0}$ sowie die Austrittsarbeit $A_k$ lässt sich durch
eine lineare Ausgleichsrechnung an $U_g$ aufgetragen gegen die Frequenz $f = \frac{c}{\lambda}$
nach der Gleichung \eqref{eqn:ausgleichb} bestimmen.
Es ergeben sich folgende Parameter:
\begin{align}
  \frac{h}{e_0} = & -2.7315918777824717 \cdot 10^{-15} \\
  A_k           = & 1.0791957852628055 \cdot 10^{-19}.  \\
  \label{eqn:ergebnisseb}
\end{align}
Die gegeneinander aufgetragenen Werte sowie der Fit sind in folgender Abbildung wiedergegeben:
\begin{figure}[H]
  \centering
  \includegraphics[scale=0.8]{build/plotb2.pdf}
  \label{fig:plotb2}
  \caption{Die Werte für Grenzspannungen und Frequenzen sowie ein linearer Fit.}
\end{figure}
\noindent

\subsection{Untersuchung des Kurvenverlaufs für $\lambda = 578 \si{\nano\meter}$}
In folgender Abbildung ist der Photostrom für $\lambda = 578 \si{\nano\meter}$ in
Abhängigkeit zu der angelegten Spannung dargestellt.
\begin{figure}[H]
  \centering
  \includegraphics[scale=0.8]{build/plotgelbc.pdf}
  \label{fig:plotgelbc}
  \caption{Photostrom aufgetragen gegen die angelegte Spannung für $\lambda = 578 \si{\nano\meter}$.}
\end{figure}
\noindent
