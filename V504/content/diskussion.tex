\section{Diskussion}
\label{sec:Diskussion}
Im Folgenden sind die Ergebnisse aus Kapitel \ref{sec:Auswertung} noch einmal aufgelistet.
\begin{table}[H]
    \centering
    \label{tab:diss}
    \sisetup{table-format=4.3}
    \begin{tabular}{S S @{${}\pm{}$} S[table-format=3.3] S[table-format=4.2] S[table-format=2.2]}
    \toprule
    {Größe} & \multicolumn{2}{c}{Ergebnis} & {Vergleichswert} & {Abweichung [\%]} \\
    \midrule
    $a   $                      & 1.480   &  0.020  & 1.5     & 1.35 \\
    $T [\si{\kelvin}]$           & 3736.62 &  152.79 & 2237.47 & 67.0 \\
    $\Phi [\si{\electronvolt}]$ & 5.26    &  0.07   & 4.54    & 15.9 \\
    \bottomrule
\end{tabular}
\end{table}
\noindent
Die Veräufe der Kennlinien folgen den theoretischen Erwartungen. Die Sättigungsströme konnten deswegen für die drei niedrigen 
Heizleistungen recht präzise bestimmt werden. Ab einem Heizstrom von $I_H=\SI{2.4}{\ampere}$ konnte der Sättigungsstrom nicht mehr direkt
abgelesen werden (vgl. Abschnitt \ref{sec:kenn}), sondern wurde über den Wendepunkt geschätzt. Es ist also davon auszugehen, dass die 
Sättigungsströme für die hohen Heizleistungen deutlich ungenauer sind, als die für die niedrigen. 
\\\noindent
Der berechnete Exponent des Langmuir-Schottkyschen Gesetzes \eqref{eqn:langmuirschottky2} gleicht dem Theroriewert in einem sehr hohen Maße. Dies 
lässt auf eine gute Messung recht akkurate Messung der Kennlinien schließen.
\\\noindent
Die beiden berechneten Temperaturwerte weichen stark von einander ab. Auffällig dabei ist, dass die Berechnung durch das Anlaufstromgebiet 
eine Temperatur ergab, die größer ist, als die Schmelztemperatur $T_s$ von Wolfram ($T_s=\SI{3695}{\kelvin}$ \cite{AP03}.
Dies lässt auf einen systematischen Fehler bei der Messung schließen. Anzumerken ist, dass das Amperemeter, mit welchem die Messungen 
durchgeführt wurde, sehr störanfällig ist. So änderte sich der gemessene Wert, wenn die Experimentierenden sich der Apperatur näherten. 
\\\noindent
Die Abweichung der Austrittsarbeit des Wolframs von etwa $\num{15.9}\%$ liegt im Rahmen der Messungenauigkeit. Es sind daher bei dieser 
Messung also keine systematischen Fehler zu vermuten. 
