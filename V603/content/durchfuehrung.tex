\section{Durchführung}
\label{sec:Durchführung}
\subsection{Vorbereitungsaufgabe}
\label{sec:vorbereitung}
In folgender Tabelle sind die recherchierten Werte für die Energien der Kupfer $K_{alpha}$ und $K_{beta}$
Linien zusammengefasst.
\begin{table}[H]
    \centering
    \caption{Literaturwerte der Energien der Spektrallinien von Kupfer. \cite{AP03}.}
    \label{tab:brechungsind}
    \sisetup{table-format=1.8}
    \begin{tabular}{S S}
      \toprule
      {Spektrallinie} & {Energie [\si{\kilo\electronvolt}]} \\
      \midrule
    {$K_{\alpha}$} & 8.048 \\
    {$K_{\beta} $} & 8.907 \\
      \bottomrule
    \end{tabular}
  \end{table}
\noindent
\subsection{Notizen für Später}
Die Transmission eines Aluminium-Absorbers lässt sich aus dem gemessenen Emissionspektrum nach
\begin{equation}
	T(\lambda) = \frac{I_{Al}}{I_0}
	\label{eqn:trans}
\end{equation}
bestimmen.
%
Um die Intensität zu messen, wird mittels eines Geiger-Müller-Zählrohres die Anzahl der gestreuten Photonen aufgenommen.
Um bei der Messung der Intensität die Totzeit des Geiger-Müller-Zählrohres zu beachten, muss folgende Korrektur vorgenommen werden:
\begin{equation*}
	I = \frac{N}{1 - \tau \cdot N}.
\end{equation*}
Hierbei bezeichnet $\tau = 90 \mu \si{\second}$ die Totzeit des Geiger-Müller-Zählrohres und $N$ die Anzahl der Photonen.
