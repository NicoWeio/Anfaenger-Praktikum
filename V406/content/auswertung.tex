\section{Auswertung}
\label{sec:Auswertung}
Bei diesem Experiment wurde ein Laser mit einer Wellenlänge von $\SI{633}{\nano\meter}$
verwendet. Dieser hat einen Abstand von $\SI{9}{\centi\meter}$ von der Blende.
Die Blende hat wiederum einen Abstand von $\SI{125}{\centi\meter}$ von dem Detektor.
Es ergibt sich also ein Abstand von $\SI{134}{\centi\meter}$ zwischen Laser und Detektor.
Bei abgedunkeltem Raum wurde ein Dunkelstrom von $\SI{1.35}{\nano\ampere}$ gemessen.
Um diesen Wert wurden alle aufgenommenen Messwerte korrigiert.

\subsection{Der Einfachspalt}
Der Einfachspalt hat nach der Herstellerangabe eine Spaltbreite von $\SI{0.15}{\milli\meter}$.
Die aufgenommenen Messwerte sind in folgender Tabelle mit Dunkelstromkorrektur
dargestellt.

\begin{table}[H]
  \centering
    \caption{Daten der Messung unter Verwendung des Einfachspaltes.}
    \label{tab:einzelspalt1}
    \sisetup{table-format=2.1}
    \begin{tabular}
      {S S[table-format=1.5] S S[table-format=1.5]}
      \toprule
      {$l / \si{\milli\meter}$} & {$I / \si{\micro\ampere}$} & {$l / \si{\milli\meter}$} & {$I / \si{\micro\ampere}$} \\
      \midrule
      0.0  &  0.00115  & 25.5 &  1.14865 \\
      1.0  &  0.00265  & 26.0 &  1.14865 \\
      2.0  &  0.00365  & 26.5 &  1.10365 \\
      3.0  &  0.00265  & 27.0 &  0.92865 \\
      4.0  &  0.00065  & 27.5 &  0.74865 \\
      5.0  &  0.00115  & 28.0 &  0.56865 \\
      6.0  &  0.00415  & 28.5 &  0.39865 \\
      7.0  &  0.00665  & 29.0 &  0.24865 \\
      8.0  &  0.00665  & 29.5 &  0.14865 \\
      9.0  &  0.00365  & 30.0 &  0.06865 \\
      10.0 &  0.00215  & 31.0 &  0.01465 \\
      11.0 &  0.00565  & 32.0 &  0.03065 \\
      12.0 &  0.00915  & 33.0 &  0.05265 \\
      13.0 &  0.00965  & 34.0 &  0.04665 \\
      14.0 &  0.00765  & 35.0 &  0.02465 \\
      15.0 &  0.00565  & 36.0 &  0.00865 \\
      16.0 &  0.01015  & 37.0 &  0.00865 \\
      17.0 &  0.01375  & 38.0 &  0.01465 \\
      18.0 &  0.01615  & 39.0 &  0.01465 \\
      19.0 &  0.01315  & 40.0 &  0.00965 \\
      20.0 &  0.00915  & 41.0 &  0.00315 \\
      20.5 &  0.01355  & 42.0 &  0.00365 \\
      21.0 &  0.01965  & 43.0 &  0.00615 \\
      21.5 &  0.11865  & 44.0 &  0.00665 \\
      22.0 &  0.02065  & 45.0 &  0.00465 \\
      22.5 &  0.35865  & 46.0 &  0.00115 \\
      23.0 &  0.51865  & 47.0 &  0.00065 \\
      23.5 &  0.68865  & 48.0 &  0.00265 \\
      24.0 &  0.85865  & 49.0 &  0.00365 \\
      24.5 &  0.99865  & 50.0 &  0.00285 \\
      25.0 &  1.09865  &      &          \\
      \bottomrule
    \end{tabular}
  \end{table}
\noindent
Aus der Stellung $l$ des Detektors lässt sich der Beugungswinkel nach
\begin{equation}
  \label{eqn:phiausw}
    \phi \approx \tan{\phi} = \frac{l-l_0}{L}
\end{equation}
berechnen. $l_0 = 25 \si{\milli\meter}$ bezeichnet hierbei die Stellung des Detektors, bei der der Strahl
nicht gebeugt ist. $L = 125 \si{\centi\meter}$ meint den Abstand zwischen Beugungsspalt und Detektorblende.
Unter Verwendung von \textit{Scientific Python} wird eine Funktion der Form \eqref{eqn:intensität1}
an die Messwerte aus Tabelle \ref{tab:einzelspalt1} gefittet. Als Parameter ergeben sich
\begin{align*}
  A_0 & = \\
  b   & = .
\end{align*}

\subsection{Der Doppelspalt}
Der verwendete Doppelspalt besitzt nach den Herstellerangaben eine Spaltbreite von
$\SI{0.1}{\milli\meter}$ sowie einen Spaltabstand von $\SI{0.4}{\milli\meter}$.
In folgender Tabelle sind die aufgenommenen Messwerte dargestellt.
\begin{table}[H]
  \centering
    \caption{Daten der Messung unter Verwendung des Doppelspaltes.}
    \label{tab:doppelspalt1}
    \sisetup{table-format=2.1}
    \begin{tabular}
      {S S[table-format=1.5] S S[table-format=1.5]}
      \toprule
      {$l / \si{\milli\meter}$} & {$I / \si{\micro\ampere}$} & {$l / \si{\milli\meter}$} & {$I / \si{\micro\ampere}$} \\
      \midrule
      0.0  &    & 25.5 &   \\
      1.0  &    & 26.0 &   \\
      2.0  &    & 26.5 &   \\
      3.0  &    & 27.0 &   \\
      4.0  &    & 27.5 &   \\
      5.0  &    & 28.0 &   \\
      6.0  &    & 28.5 &   \\
      7.0  &    & 29.0 &   \\
      8.0  &    & 29.5 &   \\
      9.0  &    & 30.0 &   \\
      10.0 &    & 31.0 &   \\
      11.0 &    & 32.0 &   \\
      12.0 &    & 33.0 &   \\
      13.0 &    & 34.0 &   \\
      14.0 &    & 35.0 &   \\
      15.0 &    & 36.0 &   \\
      16.0 &    & 37.0 &   \\
      17.0 &    & 38.0 &   \\
      18.0 &    & 39.0 &   \\
      19.0 &    & 40.0 &   \\
      20.0 &    & 41.0 &   \\
      20.5 &    & 42.0 &   \\
      21.0 &    & 43.0 &   \\
      21.5 &    & 44.0 &   \\
      22.0 &    & 45.0 &   \\
      22.5 &    & 46.0 &   \\
      23.0 &    & 47.0 &   \\
      23.5 &    & 48.0 &   \\
      24.0 &    & 49.0 &   \\
      24.5 &    & 50.0 &   \\
      25.0 &    &      &   \\
      \bottomrule
    \end{tabular}
  \end{table}
\noindent
