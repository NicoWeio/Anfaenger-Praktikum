\section{Diskussion}
\label{sec:Diskussion}
Die Ergebnisse aus \ref{sec:Auswertung} sind im Folgenden zusammengefasst:
\begin{align*}
  A_{0,1} & = 120.837 \pm 2.362 \si{\ampere\per\meter}\\
  b_1     & = 0.00323 \pm 0.00031 \si{\milli\meter} \\
  A_{0,2} & =  0.799 \pm 0.01338 \si{\ampere\per\meter}\\
  b_2     & =  0.00323 \pm 0.00031 \si{\milli\meter}.
\end{align*}

\begin{itemize}
  \item \textit{Der Fit} \\
    Die Abweichungen der Berechnung der Spaltbreite des Einzelspaltes sowie des Doppelspaltes sind mit $6072.4 \si{\percent}$ sowie $222.9 \si{\percent}$ sehr hoch. Die gefittete Funktion beschreibt die
    Messwerte des Einzelspaltes trotz der hohen Abweichung recht gut, der Fit an die Messwerte
    des Doppelspaltes ergibt jedoch eine starke Abweichung zu der nach der Theorie
    erwarteten Kurve. Dies scheint an den Messwerten zu liegen; mit geringeren Abständen der Messungen
    hätten der Haupt- und die Nebenpeaks klarer aufgenommen werden können, was
    vermutlich zu einem besseren Fit geführt hätte. Jedoch scheinen auch gut passende
    Fits recht ungenaue Werte zu liefern, wie die Berechnung der Spaltbreite des
    Einzelspaltes zeigt. Daher scheint diese Methode zur Bestimmung der Spaltbreite
    nicht gut geeignet zu sein, hier liegen systematische Probleme vor.

  \item \textit{Die Umgebung} \\
    Bei der Aufnahme der Messwerte konnte keine vollständige Dunkelheit garantiert
    werden, da das Ablesen und Notieren der Messwerte zwingend eine Lichtquelle
    benötigte. Dies ließe sich durch eine digitale Aufnahme der Daten beheben.
    Alternativ könnte der Zeiger des Amperemeters fluoreszierend sein, was zumindest
    die Lichtquelle in der Nähe des Detektors eliminieren würde. Auch ist die
    Abdunkelung des Raumes nur begrenzt erfolgreich.

  \item \textit{Die Messgenauigkeit} \\
    Die Skala des Amperemeters hat aufgrund ihrer analogen Beschaffenheit eine
    begrenzte Genauigkeit. Dies könnte mit einer digitalen Anzeige gelöst werden.
    Auch die Messung der Abstände zwischen dem Detektor, der Blende und dem Laser
    sind mit einer gewissen Ungenauigkeit versehen. Außerdem war unklar, wo genau
    der Laser in dem Gehäuse montiert ist, wodurch die Messungen der Abstände
    möglicherweise einen weiteren Fehler aufweisen. Auch das händische Verstellen
    des Detektors könnte weitere Unsicherheiten erzeugt haben.


\end{itemize}
