\section{Diskussion}
\label{sec:Diskussion}
Die Ergebnisse aus \ref{sec:Auswertung} sind im Folgenden zusammengefasst:
\begin{align*}
  A_{0,1} & = 120.837 \pm 2.362 \si{\ampere\per\meter}\\
  b_1     & = 0.00323 \pm 0.00031 \si{\milli\meter} \\
  A_{0,2} & =  0.799 \pm 0.01338 \si{\ampere\per\meter}\\
  b_2     & =  0.00323 \pm 0.00031 \si{\milli\meter}.
\end{align*}

\begin{itemize}
  \item \textit{Der Fit} \\
    Die Abweichungen der Berechnung der Spaltbreite sind mit $6072.4 \si{\percent}$
    sowie $222.9 \si{\percent}$ sehr hoch. Die gefitteten Funktionen scheinen jedoch
    die Messwerte recht gut zu beschreiben. Für die Bestimmung des Spaltabstandes
    scheint ein Fit also nicht geeignet zu sein. Sowohl die Komplexität der gefitteten
    Funktion als auch die technische Umsetzung des Fits könnten diese hohen
    Unsicherheiten erzeugt haben.

  \item \textit{Die Umgebung} \\
    Bei der Aufnahme der Messwerte konnte keine vollständige Dunkelheit garantiert
    werden, da das Ablesen und Notieren der Messwerte zwingend eine Lichtquelle
    benötigte. Dies ließe sich durch eine digitale Aufnahme der Daten beheben.
    Alternativ könnte der Zeiger des Amperemeters fluoreszierend sein, was zumindest
    die Lichtquelle in der Nähe des Detektors eliminieren würde. Auch ist die
    Abdunkelung des Raumes nur begrenzt erfolgreich.

  \item \textit{Die Messgenauigkeit} \\
    Die Skala des Amperemeters hat aufgrund ihrer analogen Beschaffenheit eine
    begrenzte Genauigkeit. Dies könnte mit einer digitalen Anzeige gelöst werden.
    Auch die Messung der Abstände zwischen dem Detektor, der Blende und dem Laser
    sind mit einer gewissen Ungenauigkeit versehen. Außerdem war unklar, wo genau
    der Laser in dem Gehäuse montiert ist, wodurch die Messungen der Abstände
    möglicherweise einen weiteren Fehler aufweisen. Auch das händische Verstellen
    des Detektors könnte weitere Unsicherheiten erzeugt haben.


\end{itemize}
