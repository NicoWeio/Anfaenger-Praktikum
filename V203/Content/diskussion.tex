\section{Diskussion}
\label{sec:Diskussion}
\subsection{Versuch für p<1bar}
Im Folgenden sind die Ergebnisse aus Kapitel \ref{sec:auswertung} für $p<1bar$ zusammen mit den jeweiligen Literaturwerten
$L_{Lit}$ und der prozentualen Abweichung $f$ von dem Literaturwert tabellarisch dargestellt.
\begin{table}[H]
  \centering
      \caption{$L$ bei $p<1bar$ mit Literaturwert\cite{AP02} und Abweichung}
      \sisetup{table-format=5.3}
      \begin{tabular}{S S S S S}
        \toprule
        {$ L_{p<1bar} \cdot \si{\mole\per\joule}$} & {$ \Delta L_{<1bar} \cdot \si{\mole\per\joule}$} & {$ L_{Lit} \cdot \si{\mole\per\joule}$} & {$f [\%]$} & {$\Delta f [\%]$}\\
        39400.195 & 284.792 & 40657 & 3.1 & 0.7\\
        \bottomrule
      \end{tabular}
    \end{table}
\begin{table}[H]
  \centering
      \caption{$L_{i}$ bei $p>1bar$ mit Literaturwert\cite{AP03} und Abweichung}
      \sisetup{table-format=5.3}
      \begin{tabular}{S S S S S}
        \toprule
        {$ L_{i} / \si{\electronvolt}$} & {$ \Delta L_{i} / \si{\electronvolt}$} & {$ L_{Lit} / \si{\electronvolt}$} & {$f [\%]$} & {$\Delta f [\%]$}\\
        0.376 & 0.003 & 0.422 & 10.8 & 0.7\\
        \bottomrule
      \end{tabular}
    \end{table}
\noindent
Bei der Verdampfungswärme $L_{p<1bar}$, welche aus den Messwerten \ref{tab:niedringe} bei niedrigem
Druck $p<1bar$ berechnet wurde, stimmt der Messwert weitestgehend mit dem Literaturwert überein. Die
prozentuale Abweichung $f$ ist demnach vergleichsweise niedrig. Auch die Unsicherheit
$\Delta L_{p<1bar}$ deutet nicht auf einen systematischen Fehler hin, sondern liegt im Rahmen einer
Messunsicherheit.\\
Folgende Fehlerquellen können zu Abweichungen von den Messwerten \ref{tab:niedringe} führen, aus denen
dann $L_{p<1bar}$ berechnet wurde:
\begin{enumerate}
  \item \textit{Menschlicher Fehler}\\
    Da das Ablesen der Temperatur und die Regulierung des Kühlwassers nicht automatisiert, sondern durch
    einen Menschen durchgeführt wurden, sind die Daten mit zusätzlichen Messunsicherheiten behaftet.
  \item \textit{Abkühlen des Wassers}\\
    Auf Grund technischer Problme mit der Messaperatur sind lag die Ausgangstemperatur des Wassers
    bei $\SI{301.65}{\kelvin}$, also über der Raumtemperatur von in etwa $\SI{293}{\kelvin}$.
\end{enumerate}

\subsection{Versuch für p<1bar}
Zunächst ist anzumerken, dass nicht beide Lösungen von Gleichung \ref{eqn:Clausius3} in Tabelle \ref{tab:L}
physikalisch sinnvoll sind. Deswegen wird im Folgenden nur die Lösung mit dem positiven Wurzelterm $L_1$
betrachtet. $L_2$, aus der Berechnung mit dem negativen Wurzelterm, wird im weiteren nicht näher beleuchtet.
Im Folgenden sind also nur die sinnvollen Ergebnisse aus Kapitel \ref{sec:auswertung} für $p>1bar$ zusammen 
mit den jeweiligen Literaturwerten $L_{Lit}$ und der prozentualen Abweichung $f$ von dem Literaturwert 
tabellarisch dargestellt.
\begin{table}[H]
  \centering
      \caption{$L$ bei $p<1bar$ mit Literaturwert\cite{AP04} und Abweichung}
      \sisetup{table-format=5.2}
      \begin{tabular}{S S S S}
        \toprule
        {$p / \si{\pascal}$} & {$ L_{p>1bar} \cdot \si{\mole\per\joule}$} & {$ L_{Lit} \cdot \si{\mole\per\joule}$} & {$f [\%]$} \\
        \midrule
        50000.00 &   53692.43 &   41525.22 &     -29.30 \\
        100000.00 &   75364.64 &   40678.50 &     -85.27 \\
        150000.00 &   67826.04 &   40102.01 &     -69.13 \\
        200000.00 &   66219.19 &   39651.63 &     -67.00 \\
        250000.00 &   59842.89 &   39291.33 &     -52.31 \\
        300000.00 &   57002.17 &   38967.05 &     -46.28 \\
        350000.00 &   54896.48 &   38678.81 &     -41.93 \\
        400000.00 &   51084.88 &   38426.59 &     -32.94 \\
        450000.00 &   48084.48 &   38192.39 &     -25.90 \\
        500000.00 &   45649.91 &   37958.19 &     -20.26 \\
        550000.00 &   44004.85 &   37760.03 &     -16.54 \\
        600000.00 &   41905.56 &   37561.86 &     -11.56 \\
        650000.00 &   39791.49 &       / &       / \\
        700000.00 &   39115.31 &   37201.55 &      -5.14 \\
        750000.00 &   37955.11 &       / &       / \\
        800000.00 &   36910.54 &   36859.26 &      -0.14  \\
        \bottomrule
      \end{tabular}
    \end{table}
Folgende Fehlerquellen können zu den großen Abweichungen von den Messwerten \ref{tab:hohe} führen, aus denen
dann $L_{p>1bar}$:
\begin{enumerate}
  \item \textit{Menschlicher Fehler}\\
    Es kann wieder zu zusätzlichen Ablesefehlern bei der Messung der Temperatur gekommen sein.
  \item \textit{Druckverlust}\\
    Bei dem Versuch viel nach kurzer Zeit auf, dass die Messapparatur an einer Verschraubung undicht war,
    sodass Wasserdampf entwich und der Druck stark abfiel. Wegen dieses Problems wurde die Messung
    erneut mit einer identischen Apparatur durchgeführt. Auch hier kam es zu dem Entweichen von Wasserdampf,
    jedoch an einem sehr viel späteren Zeitpunkt und in einem geringeren Ausmaß. Trotzdem ist ein systematischer
    Fehler durch den Druckabfall wahrscheinlich, da die Behälter laufend an Druck und Flüssigkeit verloren haben, auch
    bevor das Problem den Experimentierenden auffiel.
\end{enumerate}


% Lit.werte Tab1 https://physik.cosmos-indirekt.de/Physik-Schule/Verdampfungsw%C3%A4rme
% Lit.werte Tab2 https://www.energie-lexikon.info/verdampfungswaerme_und_kondensationswaerme.html
% Lit.werte Tab3 https://www.gemu-group.com/fileadmin/user_upload/DownloadSupport/Wissensportal/GEM%C3%9C_Dampfdrucktabelle_de.pdf