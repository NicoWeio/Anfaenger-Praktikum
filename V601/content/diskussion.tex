\section{Diskussion}
\label{sec:Diskussion}
Die Ergebnisse aus Kapitel \ref{sec:Auswertung} sind im Folgenden
zusammengefasst.
\begin{table}[H]
  \centering
    \caption{Die mittlere freie Weglänge für verschiedene Temperaturen.}
    \label{tab:freieweglaengedisk}
    \sisetup{table-format=3.2}
      \begin{tabular}{S S[table-format=2.3] S[table-format=2.3] S[table-format=2.3]}
        \toprule
        {Temperatur [$\si{\kelvin}$]} & {$p_\text{Sätt}$ [$\si{\milli\bar}$]}  & {$\bar{w}$ [$\si{\centi\meter}$]} & {$\frac{a}{\bar{w}}$} \\
        \midrule
         297.15  &    \num{4.908e-03}      &          0.591  &  \num{1.692e-2} \\
         417.05  &              3.802      & \num{7.627e-4}  &          13.111 \\
         447.25  &             11.576      & \num{2.505e-4}  &          39.916 \\
         471.15  &             25.248      & \num{1.149e-4}  &          87.064 \\
        \bottomrule
      \end{tabular}
    \end{table}
\noindent
\begin{align*}
  K_1  & = 11 \si{\volt} -  9.012 \si{\volt} = 1.988 \si{\volt} \\
  K_2  & = 11 \si{\volt} -  8.860 \si{\volt} = 2.14  \si{\volt}
\end{align*}
\begin{align*}
  U_{11}  & = 5.078 \pm 0.171 \si{\electronvolt} \\
  U_{12}  & = 5.340 \pm 0.082 \si{\electronvolt}
\end{align*}
\noindent
Verglichen mit den Literaturwert $\SI{4.9}{\electronvolt}$ \cite{chemie.de}
ergibt sich eine Abweichung von $\SI{3.6}{\percent}$ respektive $\SI{10.2}{\percent}$.
\begin{align*}
  \lambda_1 & = 244 \pm 8  \si{\nano\meter} \\
  \lambda_2 & = 229.6 \pm 3.5 \si{\nano\meter}.
\end{align*}
\noindent
Verglichen mit dem Literaturwert (berechnet nach \cite{chemie.de}) weichen die
Werte um $\SI{3.5}{\percent}$ sowie um $\SI{9.3}{\percent}$ ab.
Hier ist gut erkennbar, dass beide Messreihen recht nahe beieinanderliegende Ergebnisse
liefern. Die Abweichungen von der Theorie sind relativ gering.
Jedoch traten während des Versuches einige Schwierigkeiten auf. Diese sind im
Folgenden aufgeführt.
\begin{itemize}
  \item \textit{Der x-y-Schreiber} \\
  Der x-y-Schreiber, mit dem die Kurven aufgenommen wurde, fiel nach der zweiten Messung
  aus. Möglicherweise hat dieser auch vorher keine exakten Messungen mehr aufgenommen.
  Dies könnte die Ergebnisse verfälscht haben.
  \item \textit{Die Skalengenauigkeit} \\
  Die Werte wurden dem Graphen entnommen, wobei das Millimeterpapier nur auf etwa
  einen Millimeter genau ablesbar ist. Außerdem wurde dies durch einen Menschen
  durchgeführt, was weitere Ungenauigkeiten erzeugt. Durch eine digitale Datenaufnahme
  und Verarbeitung könnte dies umgangen werden.
  \item \textit{Die Ablesegenauigkeit} \\
  Da das Ablesen der Zeiger ebenfalls durch einen Menschen durchgeführt wurde,
  sind auch hier Ungenauigkeiten zu erwarten. Besonders problematisch ist die Anzeige
  durch Zeiger, da sie je nach Perspektive anders wirken kann und so weitere
  Ungenauigkeiten entstehen können. Dies könnte entweder durch eine digitale Anzeige
  auf den Geräten oder sogar durch eine komplett digitale Datenaufnahme verhindert
  werden.
  \item \textit{Die Temperaturregelung}\\
  Die Temperatur des Gehäuses war nicht genau einzustellen, der reale Wert
  schwankt um den eingestellten Wert. Da dies spontan auftrat, konnten die
  Experimentierenden nicht immer sofort gegensteuern, weshalb hier
  möglicherweise weitere Ungenauigkeiten entstanden sind. Dies wäre durch ein
  Gerät, welches dazu in der Lage ist die Temperatur konstant zu halten, zu
  verhindern. Alternativ könnte das Heizgerät mit dem Thermometer gekoppelt
  werden, um eine automatische Korrektur bei Abweichung von dem eingestellten
  Wert durchzuführen.
\end{itemize}
