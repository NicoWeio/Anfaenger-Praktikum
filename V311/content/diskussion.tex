\section{Diskussion}
\subsection{Die Messungen}
  Die aufgenommenen Messwerte des Experimentes zeigen sowohl bei der Messung der Hall-Spannung als auch
  bei der magnetischen Flussdichte keine größeren Abweichungen. Die Messungen scheinen verglichen mit sich selbst
  plausibel zu sein. Ein nicht vermeidbarer Messfehler wird jedoch durch das Ablesen der Werte durch
  die Experimentierenden vom Messgerät entstanden sein, da die angezeigte Größe häufig zwischen mehreren Werten schwankte. Dort wurde
  dann nach Ermessen des Experimentierenden zumeist der Wert aufgenommen, welcher sich nach einer kurzen Zeitspanne am
  häufigsten zeigte. Dies ist natürlich nicht der Idealfall. Besser wäre es, wenn die Messung automatisiert aufgenommen
  werden würde in zuvor fest definierten Zeitintervallen. Zudem wurden nicht alle Nachkommastellen beachtet, sondern auf
  die jeweilige aus den Tabellen \ref{tab:hallspannung1}, \ref{tab:hallspannung2}, \ref{tab:flussdichte1} sowie
  \ref{tab:flussdichte2} zu entnehmende Länge gerundet.
\subsection{Der Aufbau des Experiments}
  Das Experiment wurde mit einer Silberprobe durchgeführt. Der ursprüngliche Plan war jedoch, eine Wolframprobe zu
  verwenden. Diese war jedoch defekt. Der Zustand der Wolframprobe lässt darauf schließen, das möglicherweise auch die
  Silberprobe nicht in optimalem Zustand war. Zudem konnte Aufgabenteil d) nicht aus dem Experiment heraus beantwortet
  werden, da das Experiment nur mit einer Silberprobe und nicht mit einer Kupfer- und einer Silberprobe durchgeführt
  wurde und daher keine Rückschlüsse auf Löcher- oder Elektronenleitung aus dem Vergleich mit dem Verhalten von Kupfer
  gezogen werden konnten. Das Messen beider Proben wurde dadurch verhindert, dass zwei Gruppen gleichzeitig den Versuch durchgeführt haben
  und die andere Gruppe die einzige Kupferprobe verwendete.
\label{sec:Diskussion}
