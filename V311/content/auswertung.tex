\section{Auswertung}
  \subsection{Abmessungen und elektrischer Widerstand der Probe}
    Die untersuchte Silberprobe ist $0.022 \si{\milli\meter}$ dick und $14.844 \si{\milli\meter}$ lang sowie
    $25.164\si{\milli\meter}$ breit. Der gemessene elektrische Widerstand beträgt $0.6 \si{\ohm}$.
  \subsection{Untersuchung des Hall-Effektes}
    \begin{table}[H]
      \centering
        \caption{Messung der Hall-Spannung mit konstant gehaltenem Probenstrom mit $\SI{5}{\ampere}$.}
        \label{tab:hallspannung1}
        \sisetup{table-format=1.4}
        \begin{tabular}{S S S }
          \toprule
          {$I_{Feld} /\si{\ampere}$} & {$U_{Hall} /\si{\milli\volt}$} & {$U_{Hall} / \si{\milli\volt}$, umgepoltes Magnetfeld} \\
          \midrule
          0   & 0.0871 & 0.0031 \\
          0.5 & 0.0885 & 0.0857 \\
          1.0 & 0.0897 & 0.0845 \\
          1.5 & 0.0914 & 0.0829 \\
          2.0 & 0.0926 & 0.0812 \\
          2.5 & 0.0942 & 0.0798 \\
          3.0 & 0.0957 & 0.0785 \\
          3.5 & 0.0971 & 0.0779 \\
          4.0 & 0.0989 & 0.0759 \\
          4.5 & 0.0992 & 0.0748 \\
          5.0 & 0.1002 & 0.0738 \\
          \bottomrule
        \end{tabular}
      \end{table}
      \begin{table}[H]
        \centering
          \caption{Messung der Hall-Spannung mit konstant gehaltenem Spulenstrom mit $\SI{5}{\ampere}$.}
          \label{tab:hallspannung1}
          \sisetup{table-format=1.4}
          \begin{tabular}{S S S }
            \toprule
            {$I_{q} /\si{\ampere}$} & {$U_{Hall} /\si{\milli\volt}$} & {$U_{Hall} / \si{\milli\volt}$, umgepoltes Magnetfeld} \\
            \midrule
            0   & 0.0015 & 0.0016 \\
            0.5 & 0.0115 & 0.0088 \\
            1.0 & 0.0213 & 0.0158 \\
            1.5 & 0.0311 & 0.0231 \\
            2.0 & 0.0409 & 0.0300 \\
            2.5 & 0.0507 & 0.0373 \\
            3.0 & 0.0605 & 0.0443 \\
            3.5 & 0.0703 & 0.0513 \\
            4.0 & 0.0802 & 0.0585 \\
            4.5 & 0.0900 & 0.0657 \\
            5.0 & 0.0999 & 0.0727 \\
            \bottomrule
          \end{tabular}
        \end{table}
  \subsection{Berechnung der Leitfähigkeitsparameter}
    \subsubsection{Anzahl der Ladungsträger pro Volumen}
        Für die Anzahl der Ladungsträger pro Volumen ergibt sich mit \ref{eqn:Uh2} nach umstellen nach n
        \begin{equation}
            n=-\frac{1}{U_{H} e_0}\frac{B\cdot I_q}{d}. \label{eqn:Uh21}
          \end{equation}
          Dabei bezeichnet $U_{H}$ die Hall-Spannung, $e_{0}$ die Elementarladung, $B$ die magnetische Flussdichte, $I_{q}$ den Platinenstrom und $d$ die Dicke
          der Probe. Mit Gleichung \ref{eqn:Uh21} ergibt sich dann die Anzahl der Ladungsträger pro Volumen für eine Silberprobe zu
          \begin{equation}
            n_{P} = (-1.0 \pm -0.6) \cdot 10^{25} \frac{1}{\si{\cubic\meter}}
          \end{equation}
          für konstant gehaltenen Platinenstrom und
          \begin{equation}
            n_{S} = (-1.7 \pm -1.5) \cdot 10^{25} \frac{1}{\si{\cubic\meter}}
          \end{equation}
        für konstant gehaltenen Spuhlenstrom.
        Der Fehler wurde mit Gaußscher Fehlerfortpflanzung
        \begin{equation}
          \sigma_f = \sqrt{\sum_{i=0}^{N} {\frac{\partial f}{\partial x_i} \cdot \sigma_{x_i}}}
          \label{eqn:gauss}
        \end{equation}
        und der Mittelwert durch ein arithmetisches Mittel berechnet.
    \subsubsection{Anzahl der Ladungsträger pro Atom}
        Die Zahl der Ladungsträger pro Atom berechnet sich mit Gleichung \ref{eqn:z}. Es gilt
        \begin{equation*}
          \rho = \frac{N_{A} m_\text{Ag}}{V}
        \end{equation*}
        Mit $a = \frac{N_{A}}{V}$ lässt sich $a$ nun schreiben als
        \begin{equation*}
          a = \frac{\rho}{m_\text{Ag}}.
        \end{equation*}
        $m_\text{Ag}$ bezeichnet hierbei die Masse eines einzelnen Silberatoms und beträgt $m_\text{Ag} = 107.8682u$ mit
        der Atomaren Masseinheit $u = 1.661 \cdot 10^{-27}$. Dann lässt sich $z$ schreiben als
        \begin{equation}
          z =\frac{n V}{N_A} = \frac{n}{a} = \frac{n m_\text{Ag}}{\rho}
          \label{eqn:zzz}
        \end{equation}
        Einsetzen in \ref{eqn:zzz} ergibt nun

    \subsubsection{Die mittlere Flugzeit tau}

  \label{sec:Auswertung}
