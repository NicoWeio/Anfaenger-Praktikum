\section{Theorie}
\label{sec:Theorie}
\subsection{elektrische Leitfähigkeit}

Ein elektrischen Leiter besteht aus einer kristallinen Struktur, in welchem die Valenzelektronen der Atome
ein großes System bilden. Das hat zur Folge, dass sie dem Pauli-Prinzip unterliegen. Demnach muss jedes 
Valenzelektron einen anderen Quantenzustand beitzen. Es gibt nun so viele Elektronen verschiedener Energien,
dass man von quasikontinuierlichen Energiebändern spricht, in denen es jedoch verbotene Zonen gibt.