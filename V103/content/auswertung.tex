\section{Auswertung}
\label{sec:Auswertung}
\subsection{Messdaten}
\label{sec:mess}
In den folgenden Tabellen \ref{tab:abmessungen}, \ref{tab:einseitig} und \ref{tab:zweiseitig} sind die nach Kapitel
\ref{sec:Durchführung} aufgenommenen Messdaten aufgelistet.

\begin{table}[H]
    \centering
        \caption{Abmessungen und Gewicht der Proben, wobei $d$ bei Stab 1 und 2 die Dicke und bei Stab 3 den Durchmesser beschreibt.}
        \label{tab:abmessungen}
        \sisetup{table-format=3.1}
        \begin{tabular}{S S[table-format=1.1] S[table-format=2.1] S}
          \toprule
          {Probe} & {$d/\;\si{\centi\metre}$} & {$L /\;\si{\centi\metre}$} & {$m /\;\si{\gram}$}\\
          \midrule
          $\text{Stab 1}$ & 1.2 & 59.3 & 163.1\\
          $\text{Stab 2}$ & 1.1 & 59.3 & 454.6\\
          $\text{Stab 3}$ & 1.0 & 60.2 & 394.1\\
          \bottomrule
       \end{tabular}
    \end{table}

\begin{table}[H]
    \centering
        \caption{Messdaten bei einseitiger Einspannung.}
        \label{tab:einseitig}
        \sisetup{table-format=1.3}
        \begin{tabular}{S[table-format=2.0] S S S}
          \toprule
          &
          \multicolumn{1}{c}{$m_1=\SI{800}{\gram}$} &
          \multicolumn{1}{c}{$m_2=\SI{1250}{\gram}$} &
          \multicolumn{1}{c}{$m_3=\SI{550}{\gram}$} \\
          %\cmidrule(lr){2}\cmidrule(lr){3}\cmidrule(lr){4}
          {$x/\;\si{\centi\metre}$} & {$D_1/\;\si{\milli\metre}$} & {$D_2/\;\si{\milli\metre}$} & {$D_3/\;\si{\milli\metre}$} \\
          \midrule
          5  & 0.099 & 0.050 & 0.085 \\
          10 & 0.360 & 0.180 & 0.317 \\
          15 & 0.781 & 0.400 & 0.730 \\
          20 & 1.321 & 0.799 & 1.210 \\
          25 & 2.152 & 1.100 & 1.900 \\
          30 & 2.860 & 1.525 & 2.600 \\
          35 & 3.350 & 1.900 & 3.280 \\
          40 & 4.155 & 2.333 & 4.145 \\
          45 & 5.030 & 2.805 & 4.875 \\
          50 & 5.820 & 3.300 & 5.530 \\
          \bottomrule
       \end{tabular}
    \end{table}

\begin{table}[H]
    \centering
        \caption{Messdaten bei zweiseitiger Einspannung.}
        \label{tab:zweiseitig}
        \sisetup{table-format=2.3}
        \begin{tabular}{S S}
          \toprule
          &
          \multicolumn{1}{c}{$m_3=\SI{1550}{\gram}$} \\
          %\cmidrule(lr){2}
          {$ x/\;\si{\centi\metre}$} & {$D_3/\;\si{\milli\metre}$} \\
          \midrule
          5  & 0.045 \\
          10 & 0.150 \\
          15 & 0.295 \\
          20 & 0.420 \\
          25 & 0.510 \\
          30 & 0.570 \\
          35 & 0.515 \\
          40 & 0.455 \\
          45 & 0.330 \\
          50 & 0.175 \\
          55 & 0.000 \\
          \bottomrule
       \end{tabular}
    \end{table}

\subsection{Einseitige Einspannung}
\label{sec:einseitig}
Um die Elastizitätsmoduln $E$ der Proben zu berechnen, wird der Zusammenhang \eqref{eqn:8} zwischen $x$ und $D(x)$ betrachtet. Durch die Ersetzung
\begin{equation}
    \xi=Lx^2-\frac{x^3}{3}
    \label{eqn:xi}
\end{equation}
wird $D(x)$ zu der lineraren Funktion
\begin{equation}
    D(\xi)=\frac{mg}{2EI}\xi=:a\xi+b \;.
    \label{eqn:D(xi)}
\end{equation}
In Abbildung \ref{fig:einseitig} sind die Messdaten aus Tabelle \ref{tab:einseitig} nach der Ersetzung \eqref{eqn:xi} aufgetragen.

\begin{figure}[H]
    \centering
    \includegraphics[scale= 1.]{build/plot2-alle1.pdf}
    \caption{($\xi$-$D$)-Diagramm mit linerarer Regression für alle drei Proben.}
    \label{fig:einseitig}
\end{figure}

\noindent Mit \textit{numpy}
\cite{numpy} ist eine linerare Regression berechnet worden, welche die Form von Gleichung \eqref{eqn:D(xi)} besitzt. Die berechneten
Parameter der Geraden lauten
\begin{align*}
    a_1&=\SI[per-mode=reciprocal]{0.0543 \pm 0.0014}{\per\square\metre} \qquad
    b_1=\SI{0.18   \pm 0.08}{\milli\metre}      \\
    a_2&=\SI[per-mode=reciprocal]{0.0308 \pm 0.0006}{\per\square\metre} \qquad
    b_2=\SI{0.07  \pm 0.03}  {\milli\metre}       \\
    a_3&=\SI[per-mode=reciprocal]{0.0518 \pm 0.0011}{\per\square\metre} \qquad
    b_3=\SI{0.12   \pm 0.07}   {\milli\metre}         \;.
\end{align*}
Die Unsicherheiten sind dabei von \textit{uncertainties} \cite{uncertainties} berechnet worden.
Nach Gleichung \eqref{eqn:D(xi)} kann aus der Steigung $a$ der Geraden das Elastizitätsmodul
\begin{equation}
    E=\frac{mg}{2aI}
    \label{eqn:E}
\end{equation}
berechnet werden. Das Flächentrgheitsmoment ist dabei durch Gleichung \eqref{eqn:I1} für die eckigen Stäbe 1 und 2 und duch \eqref{eqn:I2}
für den runden Stab 3 gegeben. Durch Einsetzten ergeben sich für die Elastizitätsmoduln
\begin{align*}
    E_1&=\SI{ 74.13 \pm 1.20}{\kilo\newton\per\square\milli\metre}\\
    E_2&=\SI{189.20 \pm 3.35}{\kilo\newton\per\square\milli\metre}\\
    E_3&=\SI{106.02 \pm 2.34}{\kilo\newton\per\square\milli\metre} \;.
\end{align*}

\subsection{Runder Stab bei zweiseitiger Einspannung}   
\label{sec:zweiseitig}
Analog zu Abschnitt \ref{sec:einseitig} werden neue Argumente $\eta$ definiert, welche zu einem lineraren Zusammenhang zwischen
$\eta$ und $D(\eta)$ führen. Nach den Gleichungen \eqref{eqn:9} und \eqref{eqn:10} wird
\begin{align*}
    \eta_1&=3L^2x-4x^3 \\
    \eta_2&=4x^3-12Lx^2+9L^2x-L^3
\end{align*}
gewählt, sodass sich
\begin{equation}
    D(\eta)=\frac{mg}{48EI}=:c\eta+d
\end{equation}
ergibt. $\eta_1$ beschreibt dabei nach Gleichung \eqref{eqn:9} das Intervall $[0,L/2]$ und $\eta_2$ ist für $(L/2,L]$ gültig. In den Abbildungen
\ref{fig:me9} und \ref{fig:me10} sind die Messdaten für beide Bereiche in ein ($\eta$-$D$)-Diagramm aufgetragen.

\begin{figure}[H]
    \centering
    \includegraphics[scale= 1.]{build/plot6-me9.pdf}
    \caption{($\eta_1$-$D$)-Diagramm mit linerarer Regression für Stab 3.}
    \label{fig:einseitig}
\end{figure}

\begin{figure}[H]
    \centering
    \includegraphics[scale= 1.]{build/plot7-me10.pdf}
    \caption{($\eta_2$-$D$)-Diagramm mit linerarer Regression für Stab 3.}
    \label{fig:einseitig}
\end{figure}

\noindent Die Parameter der von
\textit{numpy} \cite{numpy} berechneten Regression mit den von \textit{uncertainties} \cite{uncertainties} berechneten Unsicherheiten lauten
\begin{align*}
    c_1&=\SI[per-mode=reciprocal]{0.00303 \pm 0.00020}{\per\square\metre} \qquad
    d_1=\SI{-0.141 \pm 0.030}{\milli\metre}\\
    c_2&=\SI[per-mode=reciprocal]{0.00344 \pm 0.00007}{\per\square\metre} \qquad
    d_2=\SI{-0.192 \pm 0.012}{\milli\metre} \;.
\end{align*}

\noindent Durch die Steigung $c$ der Geraden kann nun wieder das Elastizitätsmodul
\begin{equation*}
    E=\frac{mg}{48cI}
\end{equation*}
berechnet werden. Es ergibt sich für die jeweilige Regression
\begin{align*}
    E_1&=\SI{213.32 \pm 13.77}{\kilo\newton\per\square\milli\metre} \\
    E_2&=\SI{187.34 \pm 3.80}{\kilo\newton\per\square\milli\metre} \;.
\end{align*}
