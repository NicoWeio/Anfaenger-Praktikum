\section{Diskussion}
\label{sec:Diskussion}
\subsection{Abschätzung der Metalle}
\label{sec:metalle}
Um die Ergebnisse aus Kapitel \ref{sec:Auswertung} mit Litreaturwerten vergleichen zu können, wird nun eine Abschätzung vorgenommen,
um welche Metalle es sich bei den Proben handeln könnte. Dafür wird aus den Abmessungen das Volumen $V$ und dann mit der Masse der
Stäbe die Dichte
\begin{equation*}
    \rho=\frac{m}{V}
\end{equation*}
berechnet. Die berechneten Werte sind in Tabelle \ref{tab:metalle} aufgelistet.
Die prozentuale Abweichung $p$ eines Wertes $\beta$ bezüglich des Wertes $\alpha$ wird dabei mit
\begin{equation*}
    p(\alpha,\beta)=100\cdot\frac{\alpha-\beta}{\alpha}
\end{equation*}
berechnet.
\\\\
\noindent Aufgrund der Dichte und dem außeren Erscheinungsbild,
wie die Farbe des Metalls oder Verfärbungen die wahrscheinlich durch Oxidation herrühren, wird vermutet, dass es sich bei Stab 1 um
Aluminum, bei Stab 2 um Eisen und bei Stab 3 um Messing handelt. Es könnte sich jedoch auch um andere Metalllegierungen handeln, was
die weitere Diskussion erschwert. Zudem besteht das Problem, dass bei Messing als Legierung die Eigenschaften von der genauen Zusammensetzung
abhängen.
\begin{table}[H]
    \centering
        \caption{Vergleich der berechneten Dichten mit Literaturwerten \cite{Metalle} zu den vermuteten Metallen.}
        \label{tab:metalle}
        \sisetup{table-format=1.2}
        \begin{tabular}{S S S[table-format=1.1] S[table-format=2.2]}
          \toprule
          {Probe} &
          {$\rho_{mess}/\;\si{\gram\per\cubic\centi\metre}$} &
          {$\rho_{lit} /\;\si{\gram\per\cubic\centi\metre}$} &
          {$p /\;\si{\percent}$}\\
          \midrule
          $\text{Stab 1 / Aluminum}$ & 2.54 & 2.7 & 5.82\\
          $\text{Stab 2 / Eisen}$    & 6.82 & 7.9 & 13.64\\
          $\text{Stab 3 / Messing}$  & 8.34 & 8.4 & 0.77\\
          \bottomrule
       \end{tabular}
    \end{table}
Die Abweichung von den Literaturwerten der vermuteten Metalle liegt im Rahmen der Messungenauigkeit, sodass im Weiteren davon ausgegangen wird,
dass die Zuordnung der Metalle korrekt ist.

\subsection{Vergleich der Elastizitätsmoduln mit Literaturwerten}
\label{sec:vergleich}

Ein Vergleich der berechneten Elasizitätsmoduln mit Literaturwerten (Aluminium und Messing \cite{Elast1}; Eisen \cite{Elast2}) findet sich in
Tabelle \ref{tab:vergleich1}.
\begin{table}[H]
    \centering
        \caption{Vergleich der berechneten Elasizitätsmoduln für die einseitige Einspannung mit Literaturwerten \cite{Elast1} \cite{Elast2}.}
        \label{tab:vergleich1}
        \sisetup{table-format=1.1}
        \begin{tabular}{S S[table-format=3.2] @{${}\pm{}$} S[table-format=1.2] S[table-format=3.0] S}
          \toprule
          {Probe} &
          \multicolumn{2}{c}{$E_{mess}/\;\si{\kilo\newton\per\square\milli\metre}$} &
          {$E_{lit}/\;\si{\kilo\newton\per\square\milli\metre}$} &
          {$p /\;\si{\percent}$}\\
          \midrule
          $\text{Aluminum}$ & 74.13  & 1.96 & 70     & 5.9\\
          $\text{Eisen}$    & 189.20 & 3.51 & 196    & 3.5\\
          $\text{Messing}$  & 106.02 & 2.34 & 78$-$123 & 35.9$-$13.8\\ %DAS SIEHT SUPER HÄSSLICH AUS!!
          \bottomrule
       \end{tabular}
    \end{table}
\noindent Die geringen Abweichungen der berechneten Elasizitätsmoduln von den Literaturwerten lässt für Aluminum und Eisen nicht auf einen systematischen
Messfehler schließen. Bei Messing hängt die Abweichung von dem Litreaturwert stark von eben diesem ab, da Messing wie bereits erwähnt eine
Lergierung ist. Der berechnte Wert liegt jedoch recht mittig in dem Bereich, welchen die Literatur angibt.
\\\\
\noindent Die Abweichung zwischen der einseitigen und der zweiseitigen Methode ist deutlich signifikanter. Es ergeben sich
\begin{align*}
    p(E_{me,\eta,1},E_{me,\eta,2})&=\SI{12.12}{\percent}\\
    p(E_{me,\xi}, E_{me,\eta_1})  &=\SI{101.20}{\percent}\\
    p(E_{me,\xi}, E_{me,\eta_2})  &=\SI{76.73}{\percent} \;,
\end{align*}
wobei das $\xi$ nach Abschnitt \ref{sec:einseitig} für die einseitige und das $\eta$ nach Abschnitt \ref{sec:zweiseitig} für die zweiseitige
Methode steht. Da das berechnete Elasizitätsmodul für die einseitige Einspannung den Literaturwert in einem hohen Maße gleicht, ist davon
auszugehen, dass bei der zweiseitigen Methode Fehler aufgetreten sind.

\subsection{Fehlerquellen}
\label{sec:fehlerquellen}
Die Messuhren, mit welcher $D$ ausgemessen wurde, schienen sehr störanfällig zu sein. So änderte sich der angezeigte Wert, wenn der Tisch,
auf welchem die Apparatur stand berührt wurde. Bei dem Messvorgang viel zudem auf, dass sich die Messuhren nicht zuverlässig in ihre
Ausgangsposition zurück begaben, wenn das Gewicht vorsichtig befestigt und wieder abgenommen wurde. Wie groß der Einfluss des
Aufhängen des Gewichts ist, kann nur schwer abgeschätzt werden, da auch das Abnehmen den Fehler verursacht haben kann.
\\
\noindent
Bei den Proben fiel auf, dass die Stäbe teils ohne zusätzliche Gewichte schon verformt waren un Unebenheiten auf der Oberfläche besaßen.
Dadruch, dass bei dem Messvorgang nur
Differenzen der Anzeigen betrachtet wurden, ist der Einfluss dieses Umstandes bei den eckigen Stangen 1 und 2 wahrscheinlich nicht
allzu groß. Bei der runden Stange 3 kommt jedoch das Problem hinzu, dass die Rollen der Messuhren nicht immer über den höchsten
Punkt der Stange fuhren. Dies könnte die Messung negativ beeinflusst haben.
\\
\noindent
Da bei der zweiseitigen Methode kaum eine Biegung zu erkennen war, wurde ein recht hohes Gewicht von $m=\SI{1.55}{\kilo\gram}$
verwendet. Es ist möglich, dass die dadurch entstandene hohe Auslenkung die Näherungen aus Kapitel \ref{sec:Theorie} verletzt werden.
Dies könnte zu einer fehlerhaften Berechnung des Elasizitätsmodul führen und somit die hohen Abweichungen zwischen den beiden Methoden
erklären.
