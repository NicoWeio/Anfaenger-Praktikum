\section{Diskussion}
\label{sec:Diskussion}

\begin{itemize}
  \item \textit{Der gedämpfte Schwingkreis}
  Durch die Regression auf Basis einer e-Funktion lassen sich die vorliegenden Daten gut
  annähern. Jedoch scheint entweder beim Ablesen vom Oszilloskop ein systematischer
  Fehler aufgetreten zu sein oder eines der Geräte hat einen technischen Fehler, da der
  gemessene effektive Widerstand sehr viel geringer ist als der berechnete.
  Alternativ könnte  auch der Wert für den Widerstand $R_1$ falsch aufgeschrieben worden
  sein. Bei der Berechnung des effektiven Widerstandes wurden Faktoren wie der
  Innenwiderstand des Oszilloskops oder der Widerstand der Kabel nicht berückscihtigt,
  weshalb der errechnete effektive Widerstand höher sein sollter als der gemessene
  Widerstand. Dies ist hier nicht der Fall, der errechnete effektive Widerstand ist etwa
  um einen Faktor $10^2$ zu klein.

  \item \textit{Der aperiodische Grenzfall}
  Die Abweichung des gemessene Widerstand $R_{ap}$ von dem theoretisch errechneten Wert
  deutet auf ein Problem bei der Messung von $R_{ap}$ hin. EIn technischer Fehler des
  Oszilloskops oder auch des Messfühlers könnte hier der Grund für die Diskrepanz
  ziwschen den Werten sein. Während des Experimentes kam es zu mehreren Fehlern
  dergestalt, dass sich durch Berühren der Kabel durch die Experimentierenden die
  angezeigte Kurve grundlegend veränderte oder auch gar nicht mehr sichtbar war. Auch
  schien der Tastkopf zu Beginn nicht richtig in der Fassung zu sitzen, dies fiel leider
  erst im Laufe des Experimentes auf. Ebenfalls hatte die Verbindung zwischen
  Oszilloskop, Netzgerät und Schwingkreis einen Wackelkontakt. Dies könnte weitere
  Probleme verursacht haben.

  \item \textit{Der Serienresonanzkreis}
  Die theoretische errechneten Werte können hier nicht mit dem Experiment verglichen werden, da die durchgeführte Messreihe nicht in dem richtigen Intervall liegt. Wenn man von den errechneten Werten ausgeht, wäre ein Hochpunkt bei 
\end{itemize}
