\newpage
\section{Diskussion}
\label{sec:Diskussion}
Im Folgenden sind die Ergebnisse aus Kapitel \ref{sec:Auswertung} noch einmal aufgelistet.
    \begin{table}[H]
        \centering
        \caption{Auflistung der Ergebnisse}
        \label{tab:test}
        \begin{tabular}{c c c}
            \toprule
            \text{Größe} & \text{Wert} & \text{Einheit}\\
            \midrule
            $I_\text{Plateau}       $   & [370, 640]          & $\si{\volt}                         $\\
            $l_\text{Plateau}       $   & 270                 & $\si{\volt}                         $\\
            $a_\text{Plateau}       $   & 1.92 \pm 0.37       & $\si{\percent\per\volt\per\second}  $\\
            $t_\text{Abstand}       $   & 50                  & $\si{\micro\second}                 $\\
            $t_{\text{tot-Osz}}    $   & 160                 & $\si{\micro\second}                 $\\
            $t_{\text{tot-2Q}}    $   & 114.96 \pm 47.27    & $\si{\micro\second}                 $\\
            $p                      $   & -39.18              & $\%                                 $\\
            \bottomrule
        \end{tabular}
    \end{table}
    \begin{table}[H]
        \centering
        \caption{Die freigesetzte Ladung pro Teilchen}
        \label{tab:ladungproteilchen2}
        \begin{tabular}{c c @{${}\pm{}$} c c @{${}\pm{}$} c}
            \toprule
            $U \; [\si{\volt}]$ & 
            \multicolumn{2}{c} {$N \; [\si{\per\second}]$}   & 
            \multicolumn{2}{c} {$Q \; [\si{\giga}\symup{e_0}]$} \\ 
            \midrule
            350 & 163.95 & 12.80 & 36.64 & 7.50\\
            400 & 166.58 & 12.91 & 36.06 & 7.38\\
            450 & 171.07 & 13.08 & 35.12 & 7.17\\
            500 & 169.18 & 13.01 & 35.51 & 7.25\\
            550 & 169.73 & 13.03 & 35.39 & 7.23\\
            600 & 170.88 & 13.07 & 35.16 & 7.18\\
            650 & 174.88 & 13.22 & 34.35 & 7.00\\
            700 & 192.45 & 13.87 & 31.22 & 6.32\\    
            \bottomrule
        \end{tabular}
    \end{table}
    \begin{table}[H]
        \centering
        \caption{Die Zahl der freigesetzten Ladungen pro Teilchen}
        \label{tab:zahlproteilchen2}
        \begin{tabular}{c c @{${}\pm{}$} c c @{${}\pm{}$} c}
            \toprule
            $U \; [\si{\volt}]$ & 
            \multicolumn{2}{c}{$N \; [\si{\per\second}]$} & 
            \multicolumn{2}{c}{$Z [\si{\giga}]$} \\
            \midrule
            350 & 163.95 & 12.80 & 11.42 & 2.10\\
            400 & 166.58 & 12.91 & 14.99 & 2.20\\
            450 & 171.07 & 13.08 & 25.54 & 2.67\\
            500 & 169.18 & 13.01 & 29.51 & 2.92\\
            550 & 169.73 & 13.03 & 36.77 & 3.37\\
            600 & 170.88 & 13.07 & 47.48 & 4.07\\
            650 & 174.88 & 13.22 & 49.97 & 4.18\\
            700 & 192.45 & 13.87 & 58.38 & 4.51\\
            \bottomrule
        \end{tabular}
    \end{table}
\newpage
\noindent Die Ergebnisse entsprechen von den Größenordnungen der Theorie und auch die berechneten Unsicherheiten
sind plausibel. Bei den Graphen sind zudem keine großen Abweichungen einzelner Messpunkte von der 
Regression zu erkennen. Dies lässt darauf schließen, dass bei dem Versuch keine gravierenden systematischen
Fehler aufgetreten sind.
\\
Es gibt jedoch Fehlerquellen, die zu Ungenauigkeiten geführt haben können. 
\begin{itemize}
    \item \textit{Ablesen vom Oszilloskop}\\
        Das Ablesen des zeitlichen Abstands zwischen Primär- und Nachentladungsimpulsen und der Totzeit
        von dem Oszilloskop (vgl. Abb.\ref{fig:Osz}) kann zu zusätzlichen Messunsicherheiten geführt haben.
        Dies kann auch ein Grund für die relativ große Abweichung zwischen $t_{tot-Osz}$ und 
        $t_{tot-2Q}$ von $p=\num{-39.18}\%$ sein. Die zwei-Quellen-Methode kann dabei als 
        genauer angenommen werden. Eine ähnlich hohe Unsicherheit ist demnach auch bei $t_{Abstand}$
        zu erwarten. 
    \item \textit{Einmalige Messung der Totzeit}\\
        Bei der Messung der Totzeit ist bei beiden Methoden nur jeweils eine Messung durchgeführt worden.
        Dadurch werden die gemessenen Werte direkt zum Berechnen der Totzeit benutzt, anstatt einen 
        Mittelwert zu bilden, bei dem der Standartfehler sehr viel geringer ist. Auch dies könnte zu 
        der großen Abweichung zwischen $t_{tot-Osz}$ und $t_{tot-2Q}$ beigetragen haben.
    \item \textit{Ablesen des Plateau-Intervalls}\\
        Druch die natürlich auftretenden Schwankungen in dem Zerfallsprozess wurde das Ablesen des 
        Plateau-Intervalls (vgl. Abb.\ref{fig:Kennlinie}) erschwert. Dies hat direkte Auswirkungen auf
        die Plateau-Länge und Plateau-Steigung. Durch die große Anzahl an Messwerten führt ein kleiner
        Fehler bei dem Ablesen des Plateau-Intervalls jedoch nur zu einem noch kleineren Fehler bei der
        Plateau-Steigung.
\end{itemize}