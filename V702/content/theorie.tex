\section{Theorie}
\label{sec:Theorie}
\subsection{Grundlagen}
\label{sec:Grundlagen}
Atomkerne sind nur für bestimmte Verhältnisse ihrer Bestandteile, den Protonen und Neutronen, stabil.
Bei sogenannten Isotopen ist nun die Protonenanzahl identisch, also handelt es sich um das gleiche 
Element, die Neutronenzahl aber ist verschieden. So sind die Verhältnisse der Protronen und Neutronen
im Atomkern bei verschiedenen Isotopen unterschiedlich, weswegen einige Isotope instabile Atomkerne
besitzen. Diese zerfallen dann radioaktiv.\\
Um die verschiedenen Isotope eindeutig in Fromeln darstellenzu können, wird die Massenzahl des Elements
oben links neben das Elementensymbol geschrieben und gegebenfalls wird auch die Kernladung unten links 
hinzugefügt.\\
Der radioaktive Zerfall ist dabei nicht deterministisch, sodass dieser durch eine Wahrscheinlichkeit 
charakterisiert wird. Diese Zerfallswahrscheinlichkeit kann durch die Halbwertszeit $T$ beschrieben 
werden. Die Halbwertszeit ist dabei die Zeit, in der die Hälte der instabilen Atomkerne zerfallen sind
und kann bei verschiedenen Isotopen um 23 Größenordnungen verteilt sein.\\
\subsection{Gewinnung von Isotopen durch Neutronenbeschuss}
In dem Versuch werden aus praktischen Gründen Isotope mit einer Halbwertszeit zwischen Sekunden und 
wenigen Stunden verwendet. Da diese auf Grund des raschen Zerfalls nicht natürlich vorkommen, werden 
sie vor Beginn der Messung durch Neutronenbeschuss hergestellt. Ein Atom $\ce{^{m}_{z}A}$ wird dabei 
mit einem Neutron $\ce{^{1}_{0}}$ beschossen und reagiert mit diesem nach
\begin{equation}
    \ce{^{m}_{z}A}+\ce{^{1}_{0}n}\rightarrow
    \ce{^{m + 1}_{z}A^*}
    \rightarrow\ce{^{m + 1}_{z}A}+\gamma. \label{eqn:iso1}
\end{equation}
Das nun entstandene Isotop $\ce{^{m + 1}_{z}A}$ ist dann jenes, welches für das Experiment benötigt
wird. In diesem Fall werden als Ausgangselemente Vanadium $\ce{^{51}V}$ und die Rhodium $\ce{^{103}Rh}$
verwendet. Die Reaktionen lauten also nach \ref{eqn:iso1}:
%Doppelpunkt eher unschön.Alternativen aber auch
\begin{align*}
    \ce{^{51}V}+\ce{^{1}n}&\rightarrow\ce{^{52}V}\\
    \ce{^{103}Rh}+\ce{^{1}n}&
    \begin{cases}
        \overrightarrow{\tiny{10\%}} \ce{^{104 i}Rh}\\
        \overrightarrow{\tiny{90\%}} \ce{^{104}Rh}
    \end{cases}
\end{align*}
Rhodium ist insofern ein Spezialfall, dass er in zwei verschiedene isomere Isotope zerfällt. Bei diesen
ist die Anordnung der Kernbausteine verschieden, weswegen sich auch der Verlauf des weiteren Zerfalls
unterscheidet, wie im folgenden Kaptiel \ref{sec:Zerfallsreaktionen} näher erläutert. 
\subsection{Zerfallsreaktionen}
\label{sec:Zerfallsreaktionen}
In diesem Abschnitt werden die Zerfälle der soeben erhaltenen Isotope weiter erläutert.
\subsubsection*{Vanadium-52} 
Der beim Vanadium-52 auftretende Zerall ist ein $\beta^-$-Zerfall. Bei diesem wird im Kern ein Neutron
in ein Proton umgewandelt. Dabei entsteht ein hochenergetisches Elektron $\beta^-$, sowie ein 
Antineutrino $\bar{v}_e$. Die Reaktion ist
\begin{equation}
    \ce{^{52}_{23}V}\rightarrow\ce{^{52}_{24}Cr}+\beta^-+\bar{v}_e. \label{eqn:VZerfall}
\end{equation}
\subsubsection*{Rhodium-104 und -104i}
Auch Rhodium-104 ist ein $\beta^-$-Strahler und zerfällt demnach auf die gleiche Art und Weise wie 
Vanadium-52. Die Reaktion ist demnach analog zu \ref{eqn:VZerfall}
\begin{equation}
    \ce{^{104}_{45}Rh}\rightarrow\ce{^{104}_{46}Pd}+\beta^-+\bar{v}_e. \label{eqn:RhZerfall}
\end{equation}
Das isomere Rhodium-104i geht durch die Emission eines $\gamma$-Quantes in Rhodium-104 über.Dieser 
Vorgang wird durch
\begin{equation}
    \ce{^{104 i}_{45}Rh}\rightarrow+\gamma
\end{equation}
beschrieben. Das dabei entstehende Rhodium-104 zerfällt dann weiter nach Gleichung \ref{eqn:RhZerfall}.
\subsection{Mathematische Beschreibung}
\label{sec:mathe}
Die in \ref{sec:Grundlagen} anschaulich definierte Halbwertszeit wird mathematisch definiert durch.
\begin{equation}
    \frac{1}{2}N_0=N_0\textbf{e}^{-\lambda t}
\end{equation}
mit der 