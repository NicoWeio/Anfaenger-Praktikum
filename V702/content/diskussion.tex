\section{Diskussion}
\label{sec:Diskussion}
  Die Ergebnisse aus Abschnitt \ref{sec:Auswertung} sind in folgenden Tabellen zusammengefasst.
  \begin{table}[H]
    \centering
    \caption{Die Ergebnisse der ersten Regression der Vanadiummessung.}
    \label{tab:ergebnisse1}
    \sisetup{table-format=3.6}
    \begin{tabular}
      {S S}
      \toprule
      {Parameter} & {Wert} \\
      \midrule
      $\lambda$            & $\SI{-0.003168}{\frac{1}{\second}}$ \\
      $\frac{1}{-\lambda}$ & $\SI{315.607344}{\second}$ \\
      $T$                 & $\SI{218.762341}{\second}$ \\
      \bottomrule
    \end{tabular}
  \end{table}
  \begin{table}[H]
    \centering
    \caption{Die Ergebnisse der zweiten Regression der Vanadiummessung.}
    \label{tab:ergebnisse2}
    \sisetup{table-format=3.6}
    \begin{tabular}
      {S S}
      \toprule
      {Parameter} & {Wert} \\
      \midrule
      $\lambda$            & $\SI{-0.003848}{\frac{1}{\second}}$ \\
      $\frac{1}{-\lambda}$ & $\SI{259.871436}{\second}$ \\
      $T$                 & $\SI{180.129153}{\second}$ \\
      \bottomrule
    \end{tabular}
  \end{table}
  \begin{table}[H]
    \centering
    \caption{Die Ergebnisse der Regression des langlebigen Zerfalls der Rhodiummessung.}
    \label{tab:ergebnisse3}
    \sisetup{table-format=3.6}
    \begin{tabular}
      {S S}
      \toprule
      {Parameter} & {Wert} \\
      \midrule
      $\lambda$            & $\SI{ -0.003274}{\frac{1}{\second}}$ \\
      $\frac{1}{-\lambda}$ & $\SI{305.478582}{\second}$ \\
      $T$                 & $\SI{211.741618}{\second}$ \\
      \bottomrule
    \end{tabular}
  \end{table}
  \begin{table}[H]
    \centering
    \caption{Die Ergebnisse der Regression des kurzlebigen Zerfalls der Rhodiummessung.}
    \label{tab:ergebnisse3}
    \sisetup{table-format=3.6}
    \begin{tabular}
      {S S}
      \toprule
      {Parameter} & {Wert} \\
      \midrule
      $\lambda$            & $\SI{-0.004992}{\frac{1}{\second}}$ \\
      $\frac{1}{-\lambda}$ & $\SI{200.309745}{\second}$ \\
      $T$                 & $\SI{138.844135}{\second}$ \\
      \bottomrule
    \end{tabular}
  \end{table}
\noindent
Die Werte erscheinen nicht unplausibel und besitzen auch die richtigen Größenordnungen. Die Messwerte scheinen auch keinem systematischen Fehler zu unterliegen.
Die Größenordnung der Unsicherheiten der Messwerte scheint ebenfalls die Aussagekraft der Werte nicht zu beeinflussen. Jedoch können trotzdem mehrere Faktoren die Messung verfälscht haben.
Die sind im Folgenden dargestellt.
\begin{itemize}
  \item \textit{Wahl der Abgrenzung zwischen kurz- und langlebigen Zerfall des Rhodiums}\\
    Da die Bestimmung des Zeitpunktes $t^{*}$ durch die Experimentierenden nach Gutdünken vorgenommen wurde, ist der Zeitpunkt keine exakte Wahl.
    Daher wird an dieser Stelle weitere Unsicherheit entstanden sein. Diese kann möglicherweise durch eine erneute Durchführung mit mehr Messpunkten verringert werden.
  \item \textit{Ungenauigkeit der linearen Regression}\\
    Da das fitten einer Regression nur ein nicht exaktes Mittel ist, um die Halbwertszeit zu bestimmen, wird auch durch die Regression eine Unsicherheit entstanden sein.
  \item \textit{Das Geiger-Müller-Zählrohr}\\
    Die im Vergleich relativ hohe Tot- und Erholungszeit des Geiger-Müller-Zählrohres verhindert, das jeder eintreffende Impuls wirklich gemessen wird und erzeugt so ebenfalls eine
    Unsicherheit.
  \item \textit{Die Untergrundrate}\\
    Die Untergrundrate kann auf zwei Wegen eine Unsicherheit erzeugen. Einerseits ist die Messung des Nulleffektes ebenfalls mit Fehlern behaftet, andereseits kann die Untergrundrate
    von unbekannten Faktoren beeinflusst werden und ggf. schwanken, was eine Berücksichtigung des Nulleffektes in diesem Experiment erschwert.
\end{itemize}
