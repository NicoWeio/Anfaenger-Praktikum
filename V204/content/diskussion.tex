\section{Diskussion}
\label{sec:diskussion}
In der folgenden Tabelle \ref{tab:diss} sind die Ergebnisse aus Kapitel \ref{sec:auswertung} noch einmal zusammengefasst.
\begin{table}[H]
    \centering
    \caption{Zusammenfassung der Ergebnisse}
    \label{tab:diss}
    \sisetup{table-format=2.2} 
    \begin{tabular}{S S @{$\pm$} S S S}
        \toprule
        {Probe} &
        \multicolumn{2}{c}{$\kappa [\si{\watt\per\metre\per\kelvin}]$} &
        {$\kappa_\text{lit}   [\si{\watt\per\metre\per\kelvin}]$\cite{AP02}} &
        {$p [\%]$}\\
        \midrule
        {Messing}   & 47.42 & 12.61 & {81-113} & {41.45-58.04} \\
        {Aluminuim} & 154.4 & 45.1  & {220}    & {29.82      } \\
        {Edelstahl} & 8.64  & 1.61  & {20-21}  & {56.8-58.9  } \\
        \bottomrule
    \end{tabular}
  \end{table}
Dabei sind die Abweichungen $p$ von den Literaturwerten mit 
\begin{equation*}
    p=\frac{\kappa_\text{lit}-\kappa}{\kappa_\text{lit}}\cdot 100
\end{equation*}
berechnet worden. 
\\\\\noindent
Die berechneten Werte weichen alle vergleichsweise stark von den Literaturwerten ab. Im Folgenden sind mögliche Gründe aufgelistet. 
\begin{itemize}
    \item \textit{Zusammensetzung der Metalle}\\
    Wie in Tabelle \ref{tab:diss} zu erkennen, gibt es keinen diskreten Wert für die Wärmeleitfähigkeit von Messing und Eldelstahl. Die 
    Wärmeleitfähigkeit ist vielmehr abhängig, von der Zusammensetzung der Legierung. Da die genauen Eigenschaften der Proben nicht bekannt
    sind, kann der Fehler nur sehr grob abgeschäzt werde.
    \item \textit{Isolierung der Proben}\\
    Die Isolierung oberhalb der Stäbe erfasst nicht deren ganze Länge, sodass wahrscheinlich ein Teil der Wärme an die Umgebnung abgestrahlt wurde.
    \item \textit{Abkühlen der Proben}\\
    Vor jeder Messung wurden die Stäbe auf unter $\SI{30}{\celsius}$ abgekühlt. Es hatten jedoch nach dem Abkühlen nicht alle Stäbe die gleiche
    Temperatur, sodass die Startemperaturen von einander abweichen. Dieses Problem trat auch innerhalb einzelner Stäbe auf. Dort gab es schon 
    zu Begin der Messung eine Temperaturdifferenz zwischen dem nahen und dem fernen Thermoelement, was bei Edelstahl besonders deutlcih wird 
    (vgl.Abb.\ref{fig:stahl}). 
    \item \textit{Wechsel der Proben}\\
    Um den Abkühlvorgang zu zeitlich zu umgehen wurde teilweise die Probe gewechselt, also vor einer Messung die warme Probe gegen eine Probe
    mit Raumtmperatur ausgetauscht. Es könnte kleine Abweichungen in den Matrialweigenschaften der Proben geben.
    \item \textit{Gauß'sche Fehlerfortpflanzung}\\
    Auffällig sind die hohen Unsicherheiten der Wärmeleitfähigkeiten. Diese folgen aus der Gauß'schen Fehlerfortpflanzung \eqref{eqn:GaußFehler}, 
    da mit drei Größen gerechnet wird, welche wiederum Unsicherheiten berechnen. Gerade $\olsi{\Delta t}$ besitzt dabei recht große Unsicherheiten,
    was dem recht Fehleranfälligen Datenentnahme aus den Temperaturverläufen geschuldet sein könnte.
    \item \textit{Umschalten zwischen COOL und HEAT}\\
    Da das Umschalten zwischen COOl und HEAT nicht automatisiert, sondern von den Experimentierenden mittels Stoppuhr vorgenommen wurde, 
    spielt ein menschlicher Fehler in die Messergebnisse ein. Somit ist die Periodendauer $T$ nicht immer exakt gleich lang gewesen.   
\end{itemize}