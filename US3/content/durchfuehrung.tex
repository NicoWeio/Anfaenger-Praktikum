\section{Durchführung}
\label{sec:Durchführung}

Der Versuch ist wie folgt aufgebaut: ein Ultraschall-Doppler-Generator ist
angeschlossen an eine
Ultraschallsonde mit einer Frequenz von $\SI{2}{\mega\hertz}$. Es werden auf den
Strömungsrohren Doppler-Prismas platziert, welche drei Flächen mit verschiedenen
Winkeln $\SI{15}{\degree}, \SI{30}{\degree}, \SI{45}{\degree}$ haben. Auf diesen
wird die Ultraschallsonde platziert. Zwischen Sonde und Prisma und zwischen Prisma
und Rohr wird Doppler-Gel aufgetragen, um einen möglichst unbeeinflussten
Wellengang zu erhalten. Alle Flächen des Doppler-Prismas haben denselben Abstand zum
Strömungsrohr um Reproduzierbarkeit zu garantieren.
Die von der Sonde gemessenen Daten werden in einen Computer gespeist und dort
ausgelesen. Die Strömunsgflüssigkeit wird von einer Pumpe durch die Ströhmunsrohre
gepumpt. Diese ist auf verschiedene \textit{rpm} einstellbar, wodurch die
Flussgeschwindigkeit variiert werden kann. \\
\\
Es werden nun auf Strömungsrohren mit drei verschiedenen Durchmessern die
jeweils passenden Doppler-Prismas angebracht und für die drei verschiedenen
Winkel für jeden Rohrdurchmesser die Frequenzverschiebung gemessen. Dies wird
für verschiedene Pumpgeschwindigkeiten wiederholt. Hierbei muss der
Ultraschallgenerator das \textit{Sample Volume} auf \textit{Large} eingestellt sein.\\
\\
Zur Bestimmung des Strömungsprofils wird die Fläche des Doppler-Prismas
mit einem Winkel von $\SI{15}{\degree}$ sowie das Ströhmunsrohr mit einem Durchmesser
von $3/8$ Zoll verwendet. Das \textit{Sample Volume} des Ultraschallgenerators
wird hierbei auf \textit{Small} eingestellt. Der Regler \textit{Depth} wird nun
verwendet, um die Messtiefe zu variieren. Die Pumpe wird auf eine Maximalleistung
von $\SI{70}{\percent}$ eingestellt. Die Messtiefe wird ausgehend von $\SI{30}{\milli\meter}$
in $\SI{0.75}{\milli\meter}$ Schritten bis auf $\SI{11}{\milli\meter}$ verringert.
Bei jedem Schritt wird die Streuintensität und die Strömungsgeschwindigkeit aufgenommen.
Die Messung wird bei einer Pumpenleistung von $\SI{45}{\percent}$ wiederholt.

\section{Vorbereitungsaufgabe}
\label{sec:vorbereitung}
Die Dopplerwinkel $\alpha_{\theta}$ lassen sich mittels Formel \eqref{eqn:5} aus
den Prismenwinkel
$\theta$ berechnen. Da der Dopplerwinkel zu dem
Prismenwinkel $\theta=45°$ in Kapitel \ref{sec:Auswertung} benötigt wird, wird er
zusätzlich zu den drei gefragten Winkeln angegeben:
\begin{align*}
\alpha_{15}&\approx 80.06°\\
\alpha_{30}&\approx 70.53°\\
\alpha_{45}&\approx 61.87°\\
\alpha_{60}&\approx 54.74°  .
\end{align*}
